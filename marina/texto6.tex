\documentclass[11pt, a4paper]{article}
\usepackage[utf8]{inputenc}
\usepackage[brazil]{babel}
\usepackage{amsmath, amssymb}
\usepackage{geometry}
\usepackage{booktabs}
\usepackage{array}
\usepackage{indentfirst}
\usepackage{setspace}

% Layout
\geometry{left=3cm, right=2cm, top=3cm, bottom=2cm}
\onehalfspacing

\title{Teste de Hipótese em ANOVA para Delineamento Inteiramente Casualizado}
\author{Marina Oliveira Cunha \\ Universidade Federal de Pernambuco}
\date{\today}

\begin{document}
\maketitle

\section{Introdução}

A Análise de Variância (ANOVA), desenvolvida por Fisher, permite testar a igualdade das médias de múltiplos tratamentos. No Delineamento Inteiramente Casualizado (DIC), os tratamentos são distribuídos aleatoriamente às unidades experimentais. O objetivo é testar:
\[
H_0: \mu_1 = \cdots = \mu_t
\quad\text{versus}\quad
H_1: \text{pelo menos uma média difere}.
\]

A ANOVA de um fator pode ser vista como caso particular do modelo linear geral \(y = X\beta + \epsilon\). Assim, seu teste F decorre diretamente da teoria de testes em modelos lineares.

\section{Modelo Estatístico}

O modelo do DIC é
\[
y_{ij} = \mu + \tau_i + \epsilon_{ij}, \qquad
\epsilon_{ij} \sim N(0,\sigma^2),
\]
onde $\mu$ é a média geral e $\tau_i$ o efeito do tratamento $i$. As hipóteses são:
\[
H_0: \tau_1=\cdots=\tau_t=0
\quad\text{e}\quad
H_1: \exists\, i: \tau_i\neq 0.
\]

\section{Pressupostos e Diagnósticos}

A validade do teste F depende de:  
(i) normalidade dos erros;  
(ii) homogeneidade das variâncias;  
(iii) independência das observações.  
Testes como Shapiro–Wilk, Breusch–Pagan e gráficos de resíduos podem ser utilizados para avaliar tais pressupostos.

\section{Equivalência das Formulações}

Como $\mu_i = \mu+\tau_i$ e $\sum_{i=1}^t \tau_i=0$, temos:
\[
\mu_1=\cdots=\mu_t
\;\Longleftrightarrow\;
\tau_1=\cdots=\tau_t=0.
\]
Assim, formular as hipóteses em termos de médias ou de efeitos é matematicamente equivalente.

\section{Partição da Soma de Quadrados}

A partir da decomposição
\[
y_{ij}-\bar y = (y_{ij}-\bar y_i) + (\bar y_i-\bar y),
\]
obtemos
\[
SQ_T = SQ_E + SQ_A,
\]
pois $\sum_j(y_{ij}-\bar y_i)=0$. Aqui, $SQ_A$ mede a variação entre as médias dos tratamentos e $SQ_E$ a variação dentro dos tratamentos.

\section{Distribuições das Somas de Quadrados}

Devido à normalidade e independência:
\[
\frac{SQ_E}{\sigma^2} \sim \chi^2_{t(r-1)}.
\]
Sob $H_0$,
\[
\frac{SQ_A}{\sigma^2} \sim \chi^2_{t-1}.
\]

A independência entre $SQ_A$ e $SQ_E$ decorre da ortogonalidade dos componentes e do Teorema de Cochran aplicado às projeções ortogonais no modelo linear.

\section{Estatística F}

A estatística de teste é
\[
F=\frac{MS_A}{MS_E}
= \frac{SQ_A/(t-1)}{SQ_E/[t(r-1)]}.
\]
Sob $H_0$,
\[
F\sim F_{t-1,\;t(r-1)}.
\]

\section{Esperanças dos Quadrados Médios}

\[
E[MS_E]=\sigma^2,\qquad
E[MS_A] = \sigma^2 + \frac{r}{t-1}\sum_{i=1}^t \tau_i^2.
\]
Sob $H_0$, as esperanças são iguais; sob $H_1$, $MS_A$ tende a ser maior, justificando o uso do teste F.

\section{Tabela Geral da ANOVA}

\begin{table}[h]
\centering
\begin{tabular}{lcccc}
\toprule
Fonte & GL & SQ & QM & F \\
\midrule
Tratamentos & $t-1$ & $SQ_A$ & $MS_A$ & $MS_A/MS_E$ \\
Erro & $t(r-1)$ & $SQ_E$ & $MS_E$ & \\
Total & $tr-1$ & $SQ_T$ & & \\
\bottomrule
\end{tabular}
\end{table}

\section{Comparações Múltiplas (Comentário)}

Quando o teste F rejeita $H_0$ e há \textbf{mais de dois tratamentos}, pode ser necessário identificar quais médias diferem. Métodos comuns incluem:
\begin{itemize}
\item \textbf{Tukey HSD} — controla o erro tipo I familiar;
\item \textbf{Bonferroni} — mais conservador, baseado em correção de nível.
\end{itemize}
Devido ao limite de páginas, omitimos detalhes matemáticos.

\section{Exemplo Aplicado}

Dados de produtividade (quatro variedades, cinco repetições):

\begin{table}[h]
    \centering
    \begin{tabular}{cccc}
        \toprule
        A & B & C & D \\
        \midrule
        25 & 31 & 22 & 33 \\
        26 & 25 & 26 & 29 \\
        20 & 28 & 28 & 31 \\
        23 & 27 & 25 & 34 \\
        21 & 24 & 29 & 28 \\
        \midrule
        Totais: 115 & 135 & 130 & 155 \\
        Médias: 23 & 27 & 26 & 31 \\
        \bottomrule
    \end{tabular}
\end{table}

\subsection*{Cálculos}

\[
SQT = 14587 - \frac{535^2}{20} = 275{,}75
\]
\[
SQE = \left(\tfrac{115^2}{5}+\tfrac{135^2}{5}+\tfrac{130^2}{5}+\tfrac{155^2}{5}\right) - 14311{,}25 = 163{,}75
\]
\[
SQA = 275{,}75 - 163{,}75 = 112.
\]

\subsection*{Tabela ANOVA}

\begin{table}[h]
\centering
\begin{tabular}{lcccc}
\toprule
Fonte & GL & SQ & QM & F \\
\midrule
Entre & 3 & 163,75 & 54,583 & 7,798 \\
Dentro & 16 & 112,00 & 7,000 & \\
Total & 19 & 275,75 & & \\
\bottomrule
\end{tabular}
\end{table}

\subsection*{Conclusão}

Como $F_{calc}=7{,}798 > F_{crit}=3{,}24$ ($\alpha=0{,}05$), rejeita-se $H_0$. Há diferença significativa entre as variedades; a variedade D apresentou maior produtividade média.

\begin{thebibliography}{9}
\bibitem{fisher1935}
FISHER, R. A. \textit{The design of experiments}. 1935.

\bibitem{box1978}
BOX, G. E. P. et al. \textit{Statistics for experimenters}. Wiley, 1978.

\bibitem{snedecor1989}
SNEDECOR, G. W.; COCHRAN, W. G. \textit{Statistical methods}. Iowa State, 1989.
\end{thebibliography}

\end{document}
