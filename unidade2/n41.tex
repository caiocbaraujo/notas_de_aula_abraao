\newpage

\section*{4.2 Probabilidade de erro e função poder}

\textbf{Considere testar:}
\begin{equation}
H_0: \theta \in \Theta_0 \quad \times \quad H_1: \theta \in \Theta_1
\end{equation}

A partir de $X_1, \ldots, X_n$ uma amostra de $X$ com fdp $f(x; \theta)$, podem-se cometer dois tipos de erro:

\begin{center}
\begin{tabular}{c|c|c}
Natureza da escolha & $H_0$ verdadeira & $H_1$ verdadeira \\ \hline
Não rejeitar & --- & Erro tipo II \\
Rejeitar & Erro tipo I & ---
\end{tabular}
\end{center}

Em termos objetivos, tendo observados 
\[
X = (X_1, \ldots, X_n),
\]
um teste:

\begin{enumerate}
    \item Encontraria evidências para (não) rejeitar $H_0$.
    \item Isto é feito por particionar $\mathbb{R}^n$ em dois conjuntos: $R_c \subset \mathbb{R}^n$ chamado de região crítica e seu complementar $R_c^c$ tal que 
    \begin{equation}
    R_c \cup R_c^c = \mathbb{R}^n \quad \text{e} \quad R_c \cap R_c^c = \varnothing
    \end{equation}
    \item Se $X \in R_c$, rejeita-se $H_0: \theta \in \Theta_0$.
\end{enumerate}

