\documentclass[12pt,a4paper]{article}
\usepackage[utf8]{inputenc}
\usepackage[T1]{fontenc}
\usepackage[brazil]{babel}
\usepackage{amsmath, amssymb, amsthm}
\usepackage{geometry}
\geometry{margin=2.5cm}
\usepackage{hyperref}
\hypersetup{colorlinks=true,linkcolor=blue,urlcolor=blue}
\usepackage{tikz}
\usetikzlibrary{patterns}
\usepackage{import}
\usepackage{booktabs}
\usepackage{multirow}

% Título e informações do documento
\title{Capítulo 4 - Compilação Completa\\
\large Teste de Hipótese}
\author{Curso de Inferência Estatística - PPGEST/UFPE}
\date{Novembro 2025}

\begin{document}

\maketitle

\section*{Introdução ao Capítulo 4: Teste de Hipóteses}

O Capítulo 4 apresenta uma das ferramentas mais fundamentais e amplamente utilizadas na inferência estatística: os \textbf{Testes de Hipóteses}. Esta metodologia permite que tomemos decisões informadas sobre parâmetros populacionais desconhecidos com base em evidências amostrais, fornecendo um arcabouço rigoroso para validar ou refutar afirmações científicas.

\subsection*{Contexto e Motivação}

A teoria de testes de hipóteses, desenvolvida principalmente por Neyman e Pearson, responde à seguinte questão fundamental: \emph{``Dada uma afirmação sobre um parâmetro populacional desconhecido $\theta$, como podemos usar uma amostra aleatória para decidir se devemos aceitar ou rejeitar tal afirmação?''}

Esta questão permeia praticamente todas as áreas da ciência onde dados empíricos são coletados: desde estudos clínicos que avaliam a eficácia de novos tratamentos, passando por controle de qualidade industrial, até pesquisas sociais e econômicas.

\subsection*{Estrutura do Capítulo}

O capítulo está organizado de forma a construir progressivamente os conceitos fundamentais e suas aplicações:

\paragraph{4.1 Introdução e Conceitos Fundamentais}
Apresenta a formulação básica do problema de teste de hipóteses, incluindo:
\begin{itemize}
    \item Definição de hipótese estatística (nula $H_0$ e alternativa $H_1$)
    \item Classificação das hipóteses: simples, compostas unilaterais e bilaterais
    \item Região crítica e regra de decisão
    \item Conceito de erro Tipo I ($\alpha$) e Tipo II ($\beta$)
\end{itemize}

\paragraph{4.2 Probabilidade de Erro e Função Poder}
Desenvolve a teoria quantitativa dos testes, introduzindo:
\begin{itemize}
    \item Função poder $Q_\Upsilon(\theta)$ e sua interpretação
    \item Conceitos de tamanho e nível de significância
    \item Testes aleatorizados versus não aleatorizados
    \item Função crítica ou função do teste $\psi_\Upsilon(x)$
\end{itemize}

\paragraph{4.3 Lema de Neyman-Pearson (LNP)}
Este é um dos resultados mais importantes da teoria, que estabelece:
\begin{itemize}
    \item O teste mais poderoso (MP) para hipóteses simples
    \item A razão de verossimilhança como estatística de teste
    \item Relação entre testes MP e estatísticas suficientes
\end{itemize}

\paragraph{4.4 Testes para Hipóteses Compostas Unilaterais}
Estende a teoria para o caso mais geral de hipóteses compostas:
\begin{itemize}
    \item Conceito de Teste Uniformemente Mais Poderoso (UMP)
    \item Razão de Verossimilhança Monótona (RVM)
    \item Teorema de Karlin-Rubin
    \item Aplicações para distribuições da família exponencial
\end{itemize}

\paragraph{4.5 Testes para Hipóteses Bilaterais}
Aborda testes do tipo $H_0: \theta = \theta_0$ versus $H_1: \theta \neq \theta_0$:
\begin{itemize}
    \item Testes Uniformemente Mais Poderosos Não Viesados (UMPNV)
    \item Aplicações para parâmetros de locação e escala
\end{itemize}

\paragraph{4.6 Estatísticas de Teste Clássicas}
Apresenta as principais estatísticas de teste utilizadas na prática:
\begin{itemize}
    \item Teste Z (distribuição normal com variância conhecida)
    \item Teste t de Student (variância desconhecida)
    \item Teste Qui-quadrado ($\chi^2$) para variância
    \item Teste F para comparação de variâncias
\end{itemize}

\subsection*{Objetivos de Aprendizagem}

Ao final deste capítulo, espera-se que o estudante seja capaz de:

\begin{enumerate}
    \item Formular corretamente hipóteses estatísticas para problemas práticos
    \item Compreender e interpretar os erros Tipo I e II e suas implicações
    \item Calcular a função poder de um teste e utilizá-la para comparar testes
    \item Aplicar o Lema de Neyman-Pearson para construir testes MP
    \item Utilizar o Teorema de Karlin-Rubin para obter testes UMP
    \item Selecionar e aplicar apropriadamente as estatísticas de teste clássicas
    \item Interpretar p-valores e tomar decisões estatísticas fundamentadas
\end{enumerate}

\subsection*{Filosofia do Teste de Hipóteses}

É importante destacar a filosofia subjacente ao teste de hipóteses:
\begin{itemize}
    \item \textbf{Princípio da Falsificabilidade:} Procuramos evidências contra $H_0$ (não provamos que $H_0$ é verdadeira)
    \item \textbf{Controle de Erro:} Fixamos um nível máximo aceitável para o erro Tipo I ($\alpha$)
    \item \textbf{Maximização do Poder:} Buscamos testes que maximizem a probabilidade de detectar quando $H_0$ é falsa
    \item \textbf{Decisão com Incerteza:} Reconhecemos que toda decisão estatística está sujeita a erro
\end{itemize}

\vspace{0.5cm}

\noindent Este capítulo fornece não apenas as ferramentas técnicas necessárias para realizar testes de hipóteses, mas também desenvolve a intuição estatística fundamental para aplicar essas ferramentas de forma apropriada em contextos reais.


\newpage

% ===============================================================
% CONTEÚDO DO CAPÍTULO 4
% Importando todas as páginas da pasta cap4 (não da pasta unidade2!)
% ===============================================================

\import{../../cap4/}{n39.tex}
\import{../../cap4/}{n40.tex}
\import{../../cap4/}{n41.tex}
\import{../../cap4/}{n42.tex}
\import{../../cap4/}{n43.tex}
\import{../../cap4/}{n44.tex}
\import{../../cap4/}{n45.tex}
\import{../../cap4/}{n46.tex}
\import{../../cap4/}{n47.tex}
\import{../../cap4/}{n48.tex}
\import{../../cap4/}{n49.tex}
\import{../../cap4/}{n50.tex}
\import{../../cap4/}{n51.tex}
\import{../../cap4/}{n52.tex}
\import{../../cap4/}{n53.tex}
\import{../../cap4/}{n54.tex}
\import{../../cap4/}{n55.tex}
\import{../../cap4/}{n56.tex}
\import{../../cap4/}{n57.tex}
\import{../../cap4/}{n58.tex}
\import{../../cap4/}{n59.tex}
\import{../../cap4/}{n60.tex}
\import{../../cap4/}{n61.tex}
\import{../../cap4/}{n62.tex}
\import{../../cap4/}{n63.tex}
\import{../../cap4/}{n64.tex}
\import{../../cap4/}{n65.tex}
\import{../../cap4/}{n66.tex}
\import{../../cap4/}{n67.tex}
\import{../../cap4/}{n68.tex}
\import{../../cap4/}{n69.tex}
\import{../../cap4/}{n70.tex}
\import{../../cap4/}{n71.tex}
\import{../../cap4/}{n72.tex}
\import{../../cap4/}{n73.tex}
\import{../../cap4/}{n74.tex}
\import{../../cap4/}{n75.tex}
\import{../../cap4/}{n76.tex}
\import{../../cap4/}{n77.tex}
\import{../../cap4/}{n78.tex}
\import{../../cap4/}{n79.tex}
\import{../../cap4/}{n80.tex}
\import{../../cap4/}{n81.tex}
\import{../../cap4/}{n82.tex}
\import{../../cap4/}{n83.tex}

% ===============================================================
% RESUMO E CONSOLIDAÇÃO DO CAPÍTULO 4
% ===============================================================

\newpage
\section*{Resumo e Consolidação: Capítulo 4 - Teste de Hipóteses}

\subsection*{Visão Geral}

Este capítulo apresentou a teoria fundamental de testes de hipóteses, uma das principais ferramentas da inferência estatística. A seguir, consolidamos os principais conceitos, resultados teóricos e suas conexões.

\subsection*{1. Conceitos Fundamentais e Terminologia}

\begin{table}[h!]
\centering
\caption{Terminologia Básica em Testes de Hipóteses}
\begin{tabular}{@{}ll@{}}
\toprule
\textbf{Conceito} & \textbf{Definição/Interpretação} \\ \midrule
Hipótese nula $(H_0)$ & Afirmação a ser testada (status quo) \\
Hipótese alternativa $(H_1)$ & Afirmação rival a $H_0$ \\
Região crítica $(R_c)$ & Conjunto de valores amostrais que levam à rejeição de $H_0$ \\
Estatística de teste & Função dos dados usada para decidir sobre $H_0$ \\
Erro Tipo I & Rejeitar $H_0$ quando é verdadeira \\
Erro Tipo II & Não rejeitar $H_0$ quando é falsa \\
Nível de significância $(\alpha)$ & Probabilidade máxima do Erro Tipo I \\
Tamanho do teste & Probabilidade do Erro Tipo I: $\sup_{\theta \in \Theta_0} P_\theta(X \in R_c)$ \\
Função poder $Q_\Upsilon(\theta)$ & $P_\theta(\text{Rejeitar } H_0) = P_\theta(X \in R_c)$ \\
p-valor & Menor nível $\alpha$ para o qual $H_0$ seria rejeitada \\ \bottomrule
\end{tabular}
\end{table}

\subsection*{2. Classificação de Hipóteses}

\begin{table}[h!]
\centering
\caption{Tipos de Hipóteses Estatísticas}
\begin{tabular}{@{}lll@{}}
\toprule
\textbf{Tipo} & \textbf{Forma} & \textbf{Exemplo} \\ \midrule
Simples & Especifica completamente $\theta$ & $H_0: \theta = \theta_0$ \\
Composta unilateral & Um intervalo semi-infinito & $H_0: \theta \leq \theta_0$ ou $H_0: \theta \geq \theta_0$ \\
Composta bilateral & Dois intervalos ou um intervalo & $H_0: \theta \neq \theta_0$ ou $H_0: |\theta - \theta_0| > c$ \\ \bottomrule
\end{tabular}
\end{table}

\subsection*{3. Teoremas Fundamentais e Suas Aplicações}

\begin{table}[h!]
\centering
\caption{Principais Resultados Teóricos}
\begin{tabular}{@{}p{3.5cm}p{4cm}p{6cm}@{}}
\toprule
\textbf{Teorema/Lema} & \textbf{Quando Aplicar} & \textbf{Resultado Principal} \\ \midrule
Lema de Neyman-Pearson (LNP) & 
$H_0: \theta = \theta_0$ vs $H_1: \theta = \theta_1$ (simples vs simples) & 
O teste MP rejeita $H_0$ quando $\frac{L(\theta_1; x)}{L(\theta_0; x)} > k$, onde $k$ é determinado por $\alpha$ \\ \midrule

Teorema de Karlin-Rubin & 
$H_0: \theta \leq \theta_0$ vs $H_1: \theta > \theta_0$ com RVM & 
Se $T(X)$ é suficiente com RVM, o teste UMP rejeita quando $T(X) > k$ \\ \midrule

Testes UMPNV & 
$H_0: \theta = \theta_0$ vs $H_1: \theta \neq \theta_0$ (bilateral) & 
Para famílias com RVM, testes baseados em $|T(X) - c|$ podem ser UMPNV \\ \bottomrule
\end{tabular}
\end{table}

\subsection*{4. Estatísticas de Teste Clássicas}

\begin{table}[h!]
\centering
\caption{Principais Estatísticas de Teste e Suas Condições de Uso}
\begin{tabular}{@{}p{2cm}p{4cm}p{3.5cm}p{3.5cm}@{}}
\toprule
\textbf{Teste} & \textbf{Suposições} & \textbf{Estatística} & \textbf{Distribuição sob $H_0$} \\ \midrule
Teste Z & 
$X_i \sim N(\mu, \sigma^2)$, $\sigma^2$ conhecida & 
$Z = \frac{\bar{X} - \mu_0}{\sigma/\sqrt{n}}$ & 
$N(0, 1)$ \\ \midrule

Teste t & 
$X_i \sim N(\mu, \sigma^2)$, $\sigma^2$ desconhecida & 
$T = \frac{\bar{X} - \mu_0}{S/\sqrt{n}}$ & 
$t_{n-1}$ \\ \midrule

Teste $\chi^2$ & 
$X_i \sim N(\mu, \sigma^2)$, teste para $\sigma^2$ & 
$\chi^2 = \frac{(n-1)S^2}{\sigma_0^2}$ & 
$\chi^2_{n-1}$ \\ \midrule

Teste F & 
$X_i \sim N(\mu_X, \sigma_X^2)$, $Y_j \sim N(\mu_Y, \sigma_Y^2)$ & 
$F = \frac{S_X^2}{S_Y^2}$ & 
$F_{n_1-1, n_2-1}$ \\ \bottomrule
\end{tabular}
\end{table}

\subsection*{5. Estratégias para Resolução de Problemas}

\paragraph{Etapa 1: Formulação do Problema}
\begin{itemize}
    \item Identifique o parâmetro de interesse $\theta$
    \item Formule claramente $H_0$ e $H_1$
    \item Determine o nível de significância $\alpha$ (geralmente 0.05, 0.01 ou 0.10)
    \item Classifique as hipóteses (simples, composta unilateral ou bilateral)
\end{itemize}

\paragraph{Etapa 2: Escolha do Método de Teste}
\begin{itemize}
    \item Para hipóteses simples vs simples $\rightarrow$ Use o LNP
    \item Para hipóteses compostas unilaterais com RVM $\rightarrow$ Use Karlin-Rubin
    \item Para hipóteses bilaterais $\rightarrow$ Considere testes UMPNV ou teste da razão de verossimilhança
    \item Para modelos específicos (Normal, Bernoulli, Poisson, etc.) $\rightarrow$ Use testes clássicos conhecidos
\end{itemize}

\paragraph{Etapa 3: Cálculo da Estatística de Teste}
\begin{itemize}
    \item Identifique ou construa uma estatística suficiente
    \item Determine sua distribuição sob $H_0$
    \item Calcule o valor observado da estatística (valor calculado)
\end{itemize}

\paragraph{Etapa 4: Tomada de Decisão}
\begin{itemize}
    \item \textbf{Método do Valor Crítico:} Compare a estatística calculada com o valor crítico tabelado
    \item \textbf{Método do p-valor:} Calcule $p = P_{H_0}(T \geq T_{\text{obs}})$ e rejeite se $p < \alpha$
\end{itemize}

\paragraph{Etapa 5: Interpretação}
\begin{itemize}
    \item Se rejeitamos $H_0$: Há evidência estatística significativa contra $H_0$ ao nível $\alpha$
    \item Se não rejeitamos $H_0$: Não há evidência suficiente para rejeitar $H_0$ (não significa que $H_0$ é verdadeira!)
    \item Considere sempre o poder do teste e o tamanho da amostra
\end{itemize}

\subsection*{6. Fluxograma de Decisão para Escolha do Teste}

\begin{center}
\begin{tikzpicture}[
    node distance = 1.5cm,
    every node/.style = {rectangle, draw, align=center, minimum width=3cm, minimum height=1cm},
    decision/.style = {diamond, aspect=2, draw, align=center, minimum width=2cm},
    arrow/.style = {->, thick}
]

\node (inicio) {Conhece a distribuição\\populacional?};
\node[below of=inicio] (normal) {É Normal?};
\node[below left of=normal, xshift=-2cm] (variancia) {$\sigma^2$ conhecida?};
\node[below of=variancia] (testz) {\textbf{Teste Z}};
\node[right of=testz, xshift=2cm] (testet) {\textbf{Teste t}};
\node[below right of=normal, xshift=2cm] (outros) {Bernoulli, Poisson,\\Exponencial?};
\node[below of=outros] (lnp) {Use LNP ou\\Karlin-Rubin};

\draw[arrow] (inicio) -- node[right] {Sim} (normal);
\draw[arrow] (normal) -- node[above left] {Sim} (variancia);
\draw[arrow] (variancia) -- node[right] {Sim} (testz);
\draw[arrow] (variancia) -- node[above] {Não} (testet);
\draw[arrow] (normal) -- node[above right] {Não} (outros);
\draw[arrow] (outros) -- (lnp);

\end{tikzpicture}
\end{center}

\subsection*{7. Conexões Entre os Conceitos}

\paragraph{Relação entre Estatísticas Suficientes e Testes Ótimos}
\begin{itemize}
    \item O LNP garante que testes MP dependem apenas de estatísticas suficientes
    \item A suficiência reduz a dimensionalidade do problema sem perda de informação
    \item Para família exponencial: estatística suficiente $\Rightarrow$ RVM $\Rightarrow$ teste UMP via Karlin-Rubin
\end{itemize}

\paragraph{Função Poder e Comparação de Testes}
\begin{itemize}
    \item $Q_\Upsilon(\theta) = 1 - \beta(\theta)$ para $\theta \in \Theta_1$
    \item Teste A é melhor que teste B se $Q_A(\theta) \geq Q_B(\theta)$ para todo $\theta \in \Theta_1$
    \item MP $\Rightarrow$ UMP $\Rightarrow$ UMPNV (ordem decrescente de otimalidade)
\end{itemize}

\paragraph{Trade-off entre Erros Tipo I e II}
\begin{itemize}
    \item Fixado $n$: diminuir $\alpha$ aumenta $\beta$, e vice-versa
    \item Para reduzir ambos simultaneamente: aumentar $n$ (tamanho amostral)
    \item O poder aumenta com: maior $n$, maior $\alpha$, maior distância entre $\theta_0$ e $\theta_1$
\end{itemize}

\subsection*{8. Tabela de Referência Rápida: Distribuições e RVM}

\begin{table}[h!]
\centering
\caption{Distribuições Comuns e Propriedade RVM}
\begin{tabular}{@{}lcc@{}}
\toprule
\textbf{Distribuição} & \textbf{Parâmetro} & \textbf{Possui RVM?} \\ \midrule
Normal $N(\mu, \sigma^2)$ & $\mu$ ($\sigma^2$ conhecida) & Sim \\
Normal $N(\mu, \sigma^2)$ & $\sigma^2$ ($\mu$ conhecida) & Sim \\
Bernoulli$(p)$ & $p$ & Sim \\
Poisson$(\lambda)$ & $\lambda$ & Sim \\
Exponencial$(\theta)$ & $\theta$ & Sim \\
Gamma$(\alpha, \beta)$ & $\alpha$ ou $\beta$ (outro fixo) & Sim \\
Uniforme$(0, \theta)$ & $\theta$ & Não (mas tem teste UMP) \\ \bottomrule
\end{tabular}
\end{table}

\subsection*{9. Checklist de Verificação}

Ao resolver um problema de teste de hipóteses, verifique:

\begin{enumerate}
    \item[$\square$] As hipóteses $H_0$ e $H_1$ são mutuamente exclusivas e exaustivas?
    \item[$\square$] O nível de significância $\alpha$ foi especificado?
    \item[$\square$] As suposições do modelo foram verificadas (normalidade, independência, etc.)?
    \item[$\square$] A estatística de teste tem distribuição conhecida sob $H_0$?
    \item[$\square$] A região crítica foi determinada corretamente (unilateral ou bilateral)?
    \item[$\square$] O valor calculado da estatística foi comparado corretamente com o valor crítico?
    \item[$\square$] A conclusão foi expressa no contexto do problema?
    \item[$\square$] Considerou-se o poder do teste e o tamanho amostral?
\end{enumerate}

\subsection*{10. Conclusão}

O Capítulo 4 apresentou um arcabouço completo e rigoroso para testes de hipóteses, desde os conceitos fundamentais até os métodos mais sofisticados. Os principais aprendizados são:

\begin{itemize}
    \item A teoria de Neyman-Pearson fornece critérios objetivos para construir testes ótimos
    \item O conceito de poder é fundamental para avaliar a qualidade de um teste
    \item Propriedades da distribuição (RVM, suficiência) levam a testes ótimos
    \item Testes clássicos (Z, t, $\chi^2$, F) são casos especiais de princípios gerais
    \item A interpretação correta requer compreensão tanto dos aspectos técnicos quanto filosóficos
\end{itemize}

\vspace{1cm}

\noindent \textbf{Recomendação de Estudo:} Para dominar este capítulo, é essencial:
\begin{enumerate}
    \item Resolver muitos exercícios, começando pelos exemplos do texto
    \item Entender as provas dos teoremas principais (LNP e Karlin-Rubin)
    \item Praticar a interpretação de resultados no contexto de problemas reais
    \item Comparar diferentes testes para o mesmo problema
    \item Usar simulações para visualizar funções poder e distribuições das estatísticas de teste
\end{enumerate}

\end{document}

