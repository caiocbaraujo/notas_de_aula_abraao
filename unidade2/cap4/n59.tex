\newpage

\textbf{03/11/25}

Que é a densidade da exponencial padrão.

Ainda note que $\dot{X}_i \triangleq 2 \frac{X_i}{\theta} = 2 \theta^{-1} X_i$ tem densidade

\begin{equation}
f_{\dot{X}_i}(x) = \frac{1}{2} f_{X_i}\left( \frac{x}{2} \right) = \frac{1}{2} e^{-x/2}, \quad x > 0
\end{equation}

que é a densidade de $\chi^2_2$, pois $\frac{1}{2} e^{-x/2} = \frac{1}{2^{1} \Gamma(1)} e^{-x/2} x^{1-1}$.

Assim, $\sum_{i=1}^n \dot{X}_i \sim \chi^2_{2n}$, para $\dot{X}_i$ i.i.d.

Logo

\begin{equation}
R_c = \left\{ x \in \mathbb{R}^n : \frac{2}{\theta_0} \sum_{i=1}^n x_i > k_{3} \right\}
\end{equation}

Definimos a função $Q : \mathcal{X} \to \mathbb{R_+}$

\begin{equation}
Q(x) = \frac{2}{\theta_0} \sum_{i=1}^n x_i
\end{equation}

Note que para $X = (x_1, \dots, x_n)^T$, $Q(X) \overset{H_0}{\sim} \chi^2_{2n}$.

A região crítica fica definida como:

\begin{equation}
R_c = \{ x \in \mathcal{X} : Q(x) > q_{\alpha} \}
\end{equation}

em que $q_{\alpha}$ é tal que

\begin{equation}
P(Q > q_{\alpha}) = \alpha, \quad Q \sim \chi^2_{2n}
\end{equation}

\begin{center}
\begin{tikzpicture}[scale=1]
\draw[->] (-0.5,0) -- (6,0) node[right] {};
\draw[->] (0,-0.5) -- (0,2.5) node[above] {};
\draw[domain=0:5,smooth,variable=\x,black] plot ({\x},{2*exp(-(\x-2)^2)});
\draw[dashed] (4,0) -- (4,1.2);
\draw[->] (4,0.2) -- (4.5,0.8) node[right] {$\alpha$};
\node at (4,-0.3) {$q_{\alpha}$};
\end{tikzpicture}
\end{center}

