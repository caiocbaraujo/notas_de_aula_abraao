\[
R_c = \left\{ x \in \mathbb{R}^n : L(\theta_1; x) > k \, L(\theta_0; x) \right\}
\]
e
\[
R_c^c = \left\{ x \in \mathbb{R}^n : L(\theta_1; x) < k \, L(\theta_0; x) \right\}
\]

ou equivalentemente, usando a função crítica
\[
\psi(x) = 
\begin{cases}
1, & \text{se } L(\theta_1; x) > k \, L(\theta_0; x) \\
0, & \text{se } L(\theta_1; x) < k \, L(\theta_0; x)
\end{cases}
\]
em que $k \geq 0$ é determinado por
\begin{equation}
E_{\theta_0} \left[ \psi(x) \right] = \alpha \tag{4.3.2}
\end{equation}

Qualquer teste satisfazendo (4.3.1) e (4.3.2) é um teste MP de nível \(\alpha\).

\textbf{Prova:} Consideremos a prova para o caso contínuo. Note que qualquer teste \(\Upsilon\) que satisfaça (4.3.2) tem tamanho \(\alpha\) e portanto nível \(\alpha\).

Seja \(\Upsilon^*\) um teste com função crítica \(\psi_{\Upsilon^*}(x)\) e nível \(\alpha\). Sejam \(Q_{\Upsilon}(\theta)\) e \(Q_{\Upsilon^*}(\theta)\) as funções poder de \(\Upsilon\) e \(\Upsilon^*\), respectivamente.

