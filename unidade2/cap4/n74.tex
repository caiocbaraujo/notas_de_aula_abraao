

\textbf{Ex:} Sejam $x_1, \ldots, x_n$ uma amostra de $X$ tendo uma família de fdp (ou fmp)
\[
g(x; \theta) = a(\theta) \cdot c(x) \cdot e^{t(x) \cdot b(\theta)}, \quad \text{para } t \in \mathbb{X} \subset \mathbb{R} \text{ e } \theta \in \Theta.
\]
Para $\theta^*, \theta \in \Theta$ tal que $\theta^* \geq \theta$,

\begin{equation}
\frac{l(\theta^*; \mathbf{x})}{l(\theta; \mathbf{x})} 
= \frac{\prod_{i=1}^n g(x_i; \theta^*)}{\prod_{i=1}^n g(x_i; \theta)}
= \frac{a^n(\theta^*) \prod_{i=1}^n c(x_i) \cdot \exp\left\{ \sum_{i=1}^n t(x_i) b(\theta^*) \right\}}
{a^n(\theta) \prod_{i=1}^n c(x_i) \cdot \exp\left\{ \sum_{i=1}^n t(x_i) b(\theta) \right\}}
\end{equation}

\[
= \frac{a^n(\theta^*)}{a^n(\theta)} \cdot \exp\left\{ \sum_{i=1}^n t(x_i) \left[ b(\theta^*) - b(\theta) \right] \right\}.
\]

Assim $\{ g(x; \theta) : \theta \in \Theta \}$ tem RVM se $b(\theta)$ é não decrescente.

\textbf{Exemplo (Poisson):} Seja $X \sim \text{Poisson}(\lambda)$, então
\[
f(x; \lambda) = \frac{\lambda^x e^{-\lambda}}{x!} = e^{-\lambda} \cdot \frac{1}{x!} \cdot e^{x \log \lambda},
\]
ou seja,
\[
f(x; \lambda) = a(\lambda) \cdot c(x) \cdot e^{t(x) \cdot b(\lambda)},
\]
onde $a(\lambda) = e^{-\lambda}$, $c(x) = \frac{1}{x!}$, $t(x) = x$ e $b(\lambda) = \log \lambda$.

Como $b(\lambda) = \log \lambda$ é uma função não decrescente em $\lambda > 0$, a família Poisson tem RVM.

