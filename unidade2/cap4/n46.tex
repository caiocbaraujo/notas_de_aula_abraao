Seja $x = (x_1, \ldots, x_n)^T$ uma possível realização de 
$\mathbf{X} = (X_1, \ldots, X_n)^T$.

\textbf{a) $\Upsilon$ não aleatorizados:} Rejeita-se $H_0$ se e só se $x \in R_c$ ou tem função crítica
\[
\Upsilon(x) = 
\begin{cases}
1, & x \in R_c \\
0, & x \in R_c^c
\end{cases}
\]

\textbf{b) $\Upsilon$ aleatorizado:} O teste é definido por uma função crítica dada por:
\[
\Upsilon(x) = 
\begin{cases}
1, & x \in R_c \\
\delta, & x \in R_\delta \\
0, & x \in (R_c \cup R_\delta)^c
\end{cases}
\]

\textbf{Exemplo:} Sejam $X_1, \ldots, X_n$ uma amostra de $X \sim N(\theta, 25)$. \\
Neste caso $\mathbb{X} = \mathbb{R}^n$ é o espaço amostral. Considere o teste ``Rejeitar $H_0: \theta \leq 17$ se e só se $\bar{X}_n > 17 + \frac{5}{\sqrt{n}}$''. \\
$\Upsilon$ é não aleatorizado com função crítica dada por:
\[
\Upsilon(x) = 
\begin{cases}
1, & x \in \{ z \in \mathbb{R}^n : \bar{z}_n > 17 + \frac{5}{\sqrt{n}} \} \\
0, & x \in \{ z \in \mathbb{R}^n : \bar{z}_n \leq 17 + \frac{5}{\sqrt{n}} \}
\end{cases}
\]

\textbf{Exemplo:} Sejam $X_1, \ldots, X_n$ uma amostra de $X \sim \text{Bernoulli}(\theta)$ para $\theta \in (0,1)$. \\
Neste caso $\mathbb{X} = \{0,1\}^n$ é o espaço amostral.

