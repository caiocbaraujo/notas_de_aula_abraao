\noindent \textbf{29/10/25}

\vspace{0.5cm}

Vamos primeiro a verificar que
\begin{equation}
\psi_{\Upsilon}(x) - \psi_{\Upsilon^*}(x) \cdot L(\theta_1, x) - k \cdot L(\theta_1, x) \geq 0 \tag{4.3.3}
\end{equation}
\[
x \in \mathcal{X}^n
\]
Note que:

\noindent \textbf{[i]} Se $\psi_{\Upsilon}(x) = 1$, então
\begin{equation}
L(\theta_1, x) - k \cdot L(\theta_0, x) \geq 0 \quad \text{de (4.3.1)}
\end{equation}
e
\begin{equation}
\psi_{\Upsilon}(x) - \psi_{\Upsilon^*}(x) \geq 0 \quad \text{da definição da função crítica (4.3.3)}
\end{equation}
se verifica.

\noindent \textbf{[ii]} Se $\psi_{\Upsilon}(x) = 0$, então
\begin{equation}
L(\theta_1, x) - k \cdot L(\theta_0, x) \leq 0 \quad \text{de (4.3.1)}
\end{equation}
e
\begin{equation}
\psi_{\Upsilon}(x) - \psi_{\Upsilon^*}(x) \leq 0 \quad \text{da definição da função crítica (4.3.3)}
\end{equation}
se verifica.

\noindent \textbf{[iii]} Se $0 < \psi_{\Upsilon}(x) < 1$, então
\begin{equation}
L(\theta_1, x) - k \cdot L(\theta_0, x) = 0 \quad \text{e (4.3.3) se verifica}
\end{equation}

\vspace{0.5cm}

\noindent Daí, tem-se
\begin{equation}
\begin{aligned}
0 &\leq \int_{\mathcal{X}^n} \left\{ \psi_{\Upsilon}(x) - \psi_{\Upsilon^*}(x) \right\} \left[ L(\theta_1; x) - k \cdot L(\theta_0; x) \right] dx \\
&= \int_{\mathcal{X}^n} \psi_{\Upsilon}(x) \left[ L(\theta_1; x) - k \, L(\theta_0; x) \right] dx \\
&\quad - \int_{\mathcal{X}^n} \psi_{\Upsilon}^*(x) \left[ L(\theta_1; x) - k \, L(\theta_0; x) \right] dx
\end{aligned}
\end{equation}
