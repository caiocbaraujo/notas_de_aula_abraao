\section*{4.5.2 Exemplo de existência do teste UMP}

Sejam $x_1, \ldots, x_n$ uma amostra de $X \sim U(0, \theta)$, com $\theta > 0$ desconhecido. Note que $T(x) = X_{(n)}$ para $x = (x_1, \ldots, x_n)$ é suficiente para $\theta$ e tem densidade dada por

\begin{equation}
    f(t; \theta) = n \, t^{n-1} \, \theta^{-n} \, I_{(0, \theta)}(t)
\end{equation}

e para $\theta^*, \theta^* > 0$ tal que $\theta^* > \theta$:

\begin{equation}
    \frac{f(t; \theta^*)}{f(t; \theta)} = \left( \frac{\theta}{\theta^*} \right)^n \frac{I_{(0, \theta^*)}}{I_{(0, \theta)}}
\end{equation}

tem razão de verossimilhança monotônica não decrescente.

Assim, do TKR, o seguinte teste é UMP para:

\[
H_0: \theta = \theta_0 \quad \text{vs} \quad H_1: \theta > \theta_0
\]

Tal que (para $x$) como uma amostra:

\[
\psi_k(x) =
\begin{cases}
1, & T(x) > k \\
0, & T(x) \leq k
\end{cases}
\]

em que $k$ é tal que:

\begin{equation}
    \alpha = \mathbb{E}_{\theta_0} \left[ \psi_k(x) \right] = P_{\theta_0} \left( T(x) > k \right)
\end{equation}

\begin{equation}
    \alpha = \int_{k}^{\theta_0} n \, t^{n-1} \, \theta_0^{-n} \, dt 
    = \left[ \frac{t^{n}}{\theta_0^{n}} \right]_{k}^{\theta_0}
\end{equation}

\begin{equation}
    = \frac{\theta_0^{n} - k^{n}}{\theta_0^{n}} 
    = 1 - \frac{k^{n}}{\theta_0^{n}}
\end{equation}

Portanto:

\begin{equation}
    \frac{k}{\theta_0} = (1 - \alpha)^{1/n}
    \quad \Rightarrow \quad
    k = \theta_0 (1 - \alpha)^{1/n}
\end{equation}

