\newpage

\textbf{29/10/25}

Outra quantidade importante é Poder do teste.

\textbf{Definição (4.22)} O Poder ou função poder de um $\Upsilon$, denotada como $Q_\Upsilon(\theta)$, é a probabilidade de rejeitar $H_0$ quando $\theta \in \Theta$ é verdadeira.

Essa função é dada por
\begin{equation}
    Q_\Upsilon(\theta) = P_\theta[X \in R_c], \quad X \in \mathbb{R}^n, \quad \theta \in \Theta
\end{equation}

\textbf{Obs:} Note que
\begin{equation}
    \alpha = Q_\Upsilon(\theta_0)
\end{equation}
e
\begin{equation}
    1 - \beta = Q_\Upsilon(\theta_1)
\end{equation}
Para $H_0: \theta = \theta_0$ e $H_1: \theta = \theta_1$.

\textbf{Q(4.3)} Para o teste \#4 da questão Q(4.1) tem-se
\begin{equation}
    Q_\Upsilon(\theta) = P_\theta[X \in R_c]
\end{equation}
\begin{equation}
    = P_\theta[\bar{X}_n > 7.5]
\end{equation}
\begin{equation}
    \Rightarrow Q_\Upsilon(\theta) = P_\theta\left[ \frac{\bar{X}_n - \theta}{1/\sqrt{n}} > \frac{7.5 - \theta}{1/\sqrt{n}} \right], \quad Z \sim N(0,1)
\end{equation}

