\textbf{Obs 3:} Um fato interessante é que se pode obter o tamanho amostral mínimo, diga-se $n_0$, tal que
\begin{equation}
    P\left( \left| X_{n:n} - \theta \right| < \varepsilon \right) \geq 1 - \delta,
\end{equation}
em que $\varepsilon > 0$ e $\delta \in (0,1)$ são constantes pré-especificadas para $\varepsilon < \theta$.

\begin{equation}
    P\left( \left| X_{n:n} - \theta \right| < \varepsilon \right) \geq 1 - \delta
\end{equation}

\[
\therefore 1 - \left( \frac{\theta - \varepsilon}{\theta} \right)^n \geq 1 - \delta
\]
\[
\therefore \left( \frac{\theta - \varepsilon}{\theta} \right)^n \leq \delta
\]
\[
\therefore n \geq \frac{\log \delta}{\log \left( \frac{\theta - \varepsilon}{\theta} \right)}
\]
Assim:
\begin{equation}
    n_0 = \left\lceil \frac{\log \delta}{\log \left( \frac{\theta - \varepsilon}{\theta} \right)} \right\rceil + 1
\end{equation}

Para $\theta \leq \varepsilon$:
\[
n_0 = 1
\]