\newpage

\textbf{27/10/25}

\[
+ \frac{2^3 l(\theta)}{2\theta^3} (\hat{\theta}_n - \theta)^2
\]

Em que $\theta^*$ cai em segmento formado por $\theta$ e $\hat{\theta}_n$. Assim, colocando $(\hat{\theta}_n - \theta)$ em evidência.

Por TCL:
\begin{equation}
\sqrt{n} (\hat{\theta}_n - \theta) - \frac{1}{\sqrt{n}} \frac{\partial l(\theta)}{\partial \theta} = 0
\end{equation}
\begin{equation}
\frac{1}{n} \frac{\partial^2 l(\theta)}{\partial \theta^2} (\hat{\theta}_n - \theta) + \frac{1}{n} \frac{\partial^3 l(\theta)}{\partial \theta^3} \frac{(\hat{\theta}_n - \theta)^2}{2} = 0
\end{equation}

Note que 
\[
E_\theta \frac{\partial^2 \log f(X_i, \theta)}{\partial \theta^2} = - I_{X_i}(\theta)
\]
que é $(A_1)$.

Finito pela condição $(A_2)$. Daí, pela Lei Forte (fraca) dos grandes números:
\begin{equation}
\frac{1}{n} \frac{\partial^2 l(\theta)}{\partial \theta^2} = \frac{1}{n} \sum_{i=1}^n \frac{\partial^2 \log f(X_i, \theta)}{\partial \theta^2} \xrightarrow{q.c.(P)} - I_X(\theta), \quad \theta = \theta_0, \; n \to \infty
\end{equation}

De $(A_4)$:
\begin{equation}
\frac{1}{n} \frac{\partial^3 l(\theta)}{\partial \theta^3} \leq \frac{1}{n} \sum_{i=1}^n g(X_i), \quad \forall \theta \in [\theta_0 - \varepsilon, \theta_0 + \varepsilon]
\end{equation}

Adicionalmente, de $(A_5)$:

