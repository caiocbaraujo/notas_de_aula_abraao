\section*{3.7 Estimadores Consistentes}

Consistência é uma propriedade de grandes amostras introduzida por Fisher (1922).

Seja 
\[
\{T_n = T_n(X_1, \ldots, X_n); n \geq 1\}
\]
uma sequência de estimadores para $\tau(\theta)$ tal que $\theta \in \Theta \subset \mathbb{R}^p$.

\textbf{Definição (3.7.1)}: $T_n$ é consistente no sentido fraco para $\tau(\theta)$ se, e só se
\begin{equation}
T_n \xrightarrow[n \to \infty]{P} \tau(\theta)
\end{equation}

$T_n$ é inconsistente para $\tau(\theta)$ se $T_n$ não converge em probabilidade para $\tau(\theta)$.

\textbf{Obs 1}: Dados $\varepsilon > 0$ e $\delta \in (0,1)$, $n_0 = n_0(\varepsilon, \delta, \theta)$:
\begin{equation}
P_\theta\{|T_n - \theta| > \varepsilon\} \leq \delta \ \Longleftrightarrow \ P_\theta\{|T_n - \theta| \leq \varepsilon\} \geq 1 - \delta, \quad \forall n \geq n_0
\end{equation}

\textbf{Obs 2}: $T_n$ é consistente se, e só se
\begin{equation}
\lim_{n \to \infty} P_\theta\{|T_n - \theta| > \varepsilon\} = 0
\end{equation}
ou
\begin{equation}
\lim_{n \to \infty} P_\theta\{|T_n - \theta| \leq \varepsilon\} = 1
\end{equation}