\textbf{Resultado 5P} \\
Sejam $\{U_n, n \geq 1\}$ uma sequência de v.a.'s tal que
\begin{equation}
    U_n \xrightarrow{P} u \quad \text{e} \quad g(\cdot) \ \text{uma função contínua}.
\end{equation}
Então
\begin{equation}
    g(U_n) \xrightarrow[n \to \infty]{P} g(u)
\end{equation}

\textbf{Prova:} Note que se $g(x)$ é contínua, então: dado algum $\varepsilon > 0$, existe $\delta > 0$ tal que
\begin{equation}
    |g(x) - g(u)| \geq \varepsilon \ \Rightarrow \ |x - u| \geq \delta
\end{equation}
Assim, para $n$ suficientemente grande
\begin{equation}
    0 \leq P\left( |g(U_n) - g(u)| \geq \varepsilon \right) \leq P\left( |U_n - u| \geq \delta \right) \xrightarrow{n \to \infty} 0
\end{equation}
Então,
\begin{equation}
    g(U_n) \xrightarrow[n \to \infty]{P} g(u) \quad \square
\end{equation}

\textbf{Q (extra 4)} Sejam $X_1, \ldots, X_n$ v.a.'s i.i.d. tais que $X_i \sim U(0, \theta)$ para $\theta > 0$. \\
Mostre que
\begin{equation}
    T_n^2 = X_{n:n}^2 \xrightarrow[n \to \infty]{P} \theta^2
\end{equation}

\textbf{Solução:} \\
Como $X_{n:n} \xrightarrow{P} \theta$ e $g(x) = x^2$ é contínua, então pelo resultado 5P
\begin{equation}
    X_{n:n}^2 \xrightarrow[n \to \infty]{P} \theta^2
\end{equation}