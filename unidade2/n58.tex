Note que:
\begin{equation}
\frac{L_1}{L_0} = \left( \frac{\theta_0}{\theta_1} \right)^n \exp \left[ \sum_{i=1}^n x_i \left( \frac{1}{\theta_0} - \frac{1}{\theta_1} \right) \right]
\end{equation}

Daí, a região crítica de teste é dada por: Para $\mathcal{X} = \mathbb{R}_+^n$
\begin{equation}
R_c = \left\{ x \in \mathcal{X} : \left( \frac{\theta_0}{\theta_1} \right)^n \exp \left[ \sum_{i=1}^n x_i \left( \frac{1}{\theta_0} - \frac{1}{\theta_1} \right) \right] > k \right\}
\end{equation}

\begin{equation}
= \left\{ x \in \mathcal{X} : \left( \frac{1}{\theta_0} - \frac{1}{\theta_1} \right) \sum_{i=1}^n x_i > \underbrace{\log \left[ k \cdot \left( \frac{\theta_1}{\theta_0} \right)^n \right]}_{k_1} \right\}
\end{equation}

\begin{equation}
= \left\{ x \in \mathcal{X} : \sum_{i=1}^n x_i > \underbrace{\left( \frac{\theta_1 - \theta_0}{\theta_0 \theta_1} \right)^{-1} k_1}_{k_2} \right\}
\end{equation}

\begin{equation}
= \left\{ x \in \mathcal{X} : \sum_{i=1}^n x_i > k_2 \right\}
\end{equation}

Para se ter uma estatística manipulável, requer-se que sua distribuição sob $H_0$ não dependa do parâmetro. Note que
\begin{equation}
\dot{X} \triangleq \theta^{-1} X \quad \text{tem densidade}
\end{equation}

\begin{equation}
f_{\dot{X}}(x) = \theta f(\theta x; \theta)
\end{equation}

\begin{equation}
= \theta \left[ \frac{1}{\theta} e^{-\frac{\theta x}{\theta}} \right] = e^{-x}
\end{equation}