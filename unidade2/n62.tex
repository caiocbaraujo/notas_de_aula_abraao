A função correspondente é dada por
\begin{equation}
\psi(x) =
\begin{cases}
1, & \text{se } \sum_{i=1}^n x_i > k_1, \\
\delta, & \text{se } \sum_{i=1}^n x_i = k_1, \\
0, & \text{se } \sum_{i=1}^n x_i < k_1.
\end{cases}
\end{equation}

Em que o inteiro positivo $k_1$ e $\delta \in (0,1)$ são escolhidos tais que o teste tem tamanho $\alpha$.

Note que sob $H_0$, $\sum_{i=1}^n X_i \sim \text{Binomial}(n, p_0)$.

Primeiramente, determine o menor inteiro $k_1$ tal que
\begin{equation}
P_{p_0} \left[ \sum_{i=1}^n X_i > k_1 \right] < \alpha,
\end{equation}
então
\begin{equation}
\delta = \frac{\alpha - P_{p_0} \left[ \sum_{i=1}^n X_i > k_1 \right]}{P_{p_0} \left[ \sum_{i=1}^n X_i = k_1 \right]},
\end{equation}
em que
\begin{equation}
P_{p_0} \left[ \sum_{i=1}^n X_i = k_1 \right] = \binom{n}{k_1} p_0^{k_1} (1-p_0)^{n-k_1},
\end{equation}
e
\begin{equation}
P_{p_0} \left[ \sum_{i=1}^n X_i > k_1 \right] = \sum_{j=k_1+1}^n \binom{n}{j} p_0^j (1-p_0)^{n-j}.
\end{equation}

