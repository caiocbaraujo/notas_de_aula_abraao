\newpage

\textbf{29/10/25}

\begin{equation}
P_{\theta} [Z > \sqrt{n}(7,5 - \theta)] = 1 - \Phi\left( \sqrt{n}(7,5 - \theta) \right),
\end{equation}

em que

\begin{equation}
\Phi(z) = \int_{-\infty}^{z} \frac{1}{\sqrt{2\pi}} e^{-t^2/2} \, dt \quad \text{é fda de $Z$ e}
\end{equation}

\begin{equation}
\phi(t) \quad \text{é a fdp de $Z$.}
\end{equation}

Outro conceito importante é o de função crítica ou função do teste.

\textbf{Definição (4.2.3)} A função $\psi_\Upsilon: \mathbb{\chi}^n \to [0,1]$ é chamada de função crítica ou função de teste se, e só se, $\psi_\Upsilon(x)$ representa a probabilidade com a qual $H_0$ é rejeitada quando [$X = x$]é observada.

\textbf{Obs:}
\begin{equation}
Q_{\Upsilon}(\theta) = E_{\theta} \left[ \psi_{\Upsilon}(X) \right], \quad \forall \theta \in \Theta
\end{equation}

Os testes podem ser classificados como ``aleatorizados'' e ``não aleatorizados''.

\textbf{Definição (4.2.4) Tipos de Teste} Um teste $\Upsilon$ para a hipótese $H_0$ pode ser:

