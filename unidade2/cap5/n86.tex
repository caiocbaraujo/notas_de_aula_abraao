

\noindent \textbf{12/11/2015}

\begin{itemize}
    \item Para construir IC, podem ser feitos duas abordagens:
    \begin{enumerate}
        \item Procedimento de teste de hipótese
        \item Via estatística pivotal
    \end{enumerate}
\end{itemize}

\noindent \textbf{5.1 Inversão de um procedimento de teste}

Em teste de hipóteses, a região de não rejeição de $H_0$ foi denotada como:
\[
R^c = 
\begin{cases}
\{ x \in \chi^n : T(x) > k_3 \}^c & \text{p/ } H_1 : \theta \geq \theta_0, \\
\{ x \in \chi^n : T(x) < k_3 \}^c & \text{p/ } H_1 : \theta \leq \theta_0, \\
\{ x \in \chi^n : |T(x)| > k_3 \}^c & \text{p/ } H_1 : \theta \neq \theta_0.
\end{cases}
\]

Em IC é bastante relacionado a $R^c$.

