

\noindent \textbf{Data:} 12/11/25

\vspace{0.5cm}

Vamos considerar exemplos para abordar esse conceito em IC.

\medskip

\noindent \textbf{Q(5.4)} Seja $X \sim \exp(\theta)$ com densidade
\begin{equation}
    f(x; \theta) = \frac{1}{\theta} e^{-x/\theta} I_{(0,\infty)}(x)
\end{equation}

Note que $U = X / \theta$ tem densidade
\begin{equation}
    f_U(u) = \frac{dF_X(u\theta)}{du} = \theta f_X(u\theta; \theta)
\end{equation}
\begin{equation}
    f_U(u) = e^{-u} I_{(0,\infty)}(u)
\end{equation}

Então $U$ é um pivô pela Definição (5.3.1).

\medskip

Note que é possível determinar dois pontos $a, b > 0$ tais que $a < b$ e
\begin{equation}
    P(U \leq a) = P(U > b) = \frac{\alpha}{2} \quad \text{ou equivalentemente}
\end{equation}
\begin{equation}
    P(a < U < b) = 1 - \alpha
\end{equation}

Com $\alpha \in (0,1)$ fixado, detalhando:
\begin{equation}
    \int_{0}^{a} e^{-x} \, dx = 1 - e^{-a} = \frac{\alpha}{2} \quad \Rightarrow \quad a = -\log\left(1 - \frac{\alpha}{2}\right)
\end{equation}

