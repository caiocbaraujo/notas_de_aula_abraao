\section*{Problema para duas amostras}

Se $\tau_{\Theta}$ é conhecido, podem-se obter $a < b$ tais que
\begin{equation}
P_{\Theta} \left\{ a < \frac{\hat{K}(\Theta) - K(\Theta)}{\tau} < b \right\} = 1 - \alpha
\end{equation}

Desta última identidade, obtém-se o intervalo de confiança $1 - \alpha$ para $K(\Theta)$.

Para $\tau_{\Theta}$ desconhecido, estima-se $K(\Theta)$ por $\hat{K}(\Theta)$ e trabalha-se com
\begin{equation}
P_{\Theta} \left\{ a < \frac{\hat{K}(\Theta) - K(\Theta)}{\hat{\tau}} < b \right\} = 1 - \alpha
\end{equation}

Desta relação, pode-se derivar o intervalo de confiança $1 - \alpha$ para $K(\Theta)$.

No parâmetro de escala, pode-se o pivô $\frac{\hat{K}(\Theta)}{K(\Theta)}$ cuja distribuição independe de $\Theta \in \Theta$.

Então, $a$ e $b$ tais que $a < b$ são obtidos de
\begin{equation}
P_{\Theta} \left\{ a < \frac{\hat{K}(\Theta)}{K(\Theta)} < b \right\} = 1 - \alpha
\end{equation}

