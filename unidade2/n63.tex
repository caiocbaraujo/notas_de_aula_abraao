Da discussão anterior, a probabilidade  
o erro do Tipo I é dada por:
\begin{equation}
\alpha = \delta \cdot P_{p_0} \left[ \sum_{i=1}^{n} X_i = k_1 \right] + P_{p_0} \left[ \sum_{i=1}^{n} X_i > k_1 \right]
\end{equation}

\textbf{05/11/25}

\textbf{Q(4.7)} Sejam $X_1, \ldots, X_n$ uma amostra de $X \sim \text{Poisson}(\lambda)$ para $\lambda > 0$ desconhecido. Derive teste MP para:
\begin{equation}
H_0: \lambda = \lambda_0 \quad \text{vs} \quad H_i: \lambda = \lambda_1 \ (\lambda_1 > \lambda_0)
\end{equation}

\textbf{Solução:}  
Como as hipóteses são simples, o LNP se aplica. A verossimilhança é dada por:
\begin{equation}
L_i \triangleq L(\lambda_i; x) = \prod_{k=1}^{n} \left[ e^{-\lambda_i} \frac{\lambda_i^{x_k}}{x_k!} \right]
\end{equation}
\begin{equation}
= e^{-n\lambda_i} \cdot \lambda_i^{\sum_{k=1}^{n} x_k} \cdot \prod_{k=1}^{n} \frac{1}{x_k!}
\end{equation}

O teste MP é da forma:
\[
\text{Rejeitar } H_0 \quad \text{se e somente se} \quad \frac{L_1}{L_0} > k
\]

Note que:
\begin{equation}
\frac{L_1}{L_0} = \left[ \frac{e^{-\lambda_1}}{e^{-\lambda_0}} \right]^n \cdot \left[ \frac{\lambda_1}{\lambda_0} \right]^{\sum_{k=1}^{n} x_k} = e^{-n(\lambda_1 - \lambda_0)} \cdot \left( \frac{\lambda_1}{\lambda_0} \right)^{\sum_{k=1}^{n} x_k}
\end{equation}

