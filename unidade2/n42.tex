\newpage

\section*{Q(4.1) Alguns exemplos de testes}

Sejam $X_1, \ldots, X_9$ uma amostra de $X \sim N(\theta, 1)$ para $\theta \in \mathbb{R}$ desconhecido. \\
No contexto, deseja-se testar:

\begin{equation}
H_0: \theta = 5.5 \quad \text{vs} \quad H_1: \theta = 8
\end{equation}

Seja 
\begin{equation}
\bar{X}_n = 9^{-1} \sum_{i=1}^9 X_i
\end{equation}

\subsection*{Testes}

\begin{itemize}
    \item Teste \#1: rejeitam-se $H_0$ se e só se $X_1 > 7$
    \item Teste \#2: rejeitam-se $H_0$ se e só se $\frac{X_1 + X_2}{2} > 7$
    \item Teste \#3: rejeitam-se $H_0$ se e só se $\bar{X}_n > 6$
    \item Teste \#4: rejeitam-se $H_0$ se e só se $\bar{X}_n > 7.5$
\end{itemize}

\subsection*{Suas regiões críticas}

Para teste \#1:
\begin{equation}
R_c = \{ (x_1, \ldots, x_9) \in \mathbb{R}^9 : x_1 > 7 \}
\end{equation}

Para teste \#2:
\begin{equation}
R_c = \{ (x_1, \ldots, x_9) \in \mathbb{R}^9 : \frac{x_1 + x_2}{2} > 7 \}
\end{equation}

Para teste \#3:
\begin{equation}
R_c = \{ (x_1, \ldots, x_9) \in \mathbb{R}^9 : \bar{x} > 6 \}
\end{equation}

Para teste \#4:
\begin{equation}
R_c = \{ (x_1, \ldots, x_9) \in \mathbb{R}^9 : \bar{x} > 7.5 \}
\end{equation}

