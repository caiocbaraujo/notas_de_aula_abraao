Neyman e Pearson formularam o problema de testar hipótese como se segue.

Considere que se deseja escolher entre:
\begin{equation}
H_0 : \theta \in \Theta_0 \quad \times \quad H_1 : \theta \in \Theta_1
\end{equation}
tal que $\Theta = \Theta_0 \cup \Theta_1$ e $\Theta_0 \cap \Theta_1 = \varnothing$.

Então, baseado-se em uma amostra $X_1, \ldots, X_n \in X$, deve-se tomar a decisão de rejeitar $H_0$ ou não rejeitar.

As hipóteses costumam ser classificadas como:

\textbf{1) Simples:}
\begin{equation}
H_0 : \theta = \theta_0, \quad H_1 : \mu = \mu_1
\end{equation}

\textbf{2) Composta unilateral:}
\begin{equation}
H_0 : \theta \geq \theta_0, \quad H_1 : \mu < \mu_1
\end{equation}

\textbf{3) Composta bilateral:}
\begin{equation}
H_0 : \theta \neq \theta_0 \quad \text{ou} \quad [\theta \leq \theta_0 \ \text{ou} \ \theta \geq \theta_1]
\end{equation}

\textbf{Obs:} $H_0$ é chamada de hipótese nula. \\
$H_1$ é chamada de hipótese alternativa.

