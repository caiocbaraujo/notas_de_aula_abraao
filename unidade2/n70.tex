\newpage

\section*{4.4 Teste para H composta unilateral}

Considere estudar hipóteses do tipo
\begin{equation}
H_0: \theta = \theta_0 \quad \text{vs} \quad 
\begin{cases}
H_1: \theta > \theta_0 \\
H_1: \theta < \theta_0
\end{cases}
\tag{4.4.1}
\end{equation}

No que segue, apresentam-se abordagem para deduzir teste UMP para (4.4.1).

\subsection*{4.4.1 Teste UMP via Lema de Neyman Pearson}

Inicialmente fixemos um valor arbitrário $\theta_1 \in \Theta$ tal que $\theta_1 > \theta_0$. A hipótese alternativa de (4.4.1) pode ser escrita como $H_1: \theta = \theta_1$.

Agora tem-se um problema de duas hipóteses simples e, pelo LNP, existe um teste MP tal que para
\begin{equation}
H_0: \theta = \theta_0 \quad \text{vs} \quad H_1: \theta = \theta_1
\end{equation}

Se esse teste particular não é afetado pela escolha de $\theta_1$, então diz-se que tal teste é UMP.

