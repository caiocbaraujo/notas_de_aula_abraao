\[
Q_{\Upsilon}(\theta_1) \geq Q_{\Upsilon^*}(\theta_1)
\]

O que mostra que $\Upsilon$ é no mínimo tão poderoso quanto $\Upsilon^*$.

\textbf{03/11/2025}

\textbf{Obs 1: (LNP)} No Lema de Neyman Pearson nada é dito sobre o conjunto
\[
R^* := \{ x \in \mathbb{R}^n : L(\theta_1; x) - k \, L(\theta_0; x) = 0 \}
\]
Quando $X$ é contínua, a probabilidade de $X$ pertencer a $R^*$ é zero e esse detalhe não tem importância na prática. Quando $X$ é discreta, deve-se aleatorizar o evento $X \in R^*$ tal que o teste MP tenha tamanho $\alpha \in (0,1)$.

\textbf{Obs 2: (LNP)} O teste MP proposto a partir de LNP é início.

\textbf{Q(4.4)} Sejam $X_1, \ldots, X_n$ uma amostra aleatória de $X \sim N(\mu, \sigma^2)$ com $\mu$ desconhecido e $\sigma^2 \in \mathbb{R}^+$ conhecido. Encontre o teste MP de nível $\alpha$ para
\[
H_0: \mu = \mu_0 \quad \text{vs} \quad H_1: \mu = \mu_1
\]
tal que $\mu_0$ e $\mu_1$ são conhecidos e $\mu_1 > \mu_0$.

