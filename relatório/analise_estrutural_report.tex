\documentclass[12pt,a4paper]{article}
\usepackage[utf8]{inputenc}
\usepackage[T1]{fontenc}
\usepackage[brazil]{babel}
\usepackage{amsmath, amssymb}
\usepackage{geometry}
\geometry{margin=2.5cm}
\usepackage{booktabs}
\usepackage{enumitem}

\title{Análise Estrutural e Estratégia de Otimização\\
\large Relatório sobre Teste $H_0: \beta_1 = \mathbf{0}_p$\\
\large Limite: 4 páginas + título}
\author{Análise do arquivo report.tex}
\date{\today}

\begin{document}

\maketitle

\section{Visão Geral da Estrutura e Conexões}

\subsection{Resumo da Arquitetura Teórica}

O relatório segue uma arquitetura teórica coesa que conecta três pilares fundamentais:

\textbf{1. Fundação Estatística (Seção 1)} -- Estabelece o ambiente probabilístico necessário:
\begin{itemize}
    \item Modelo normal linear como base distributiva
    \item Propriedades dos estimadores (BLUE via Gauss-Markov)
    \item Teorema de Cochran como ferramenta teórica central
\end{itemize}

\textbf{2. Teoria do Teste (Seção 2)} -- Aplica a fundação ao problema específico:
\begin{itemize}
    \item Decomposição de soma de quadrados via projeções ortogonais
    \item Estatística $F$ como razão de $\chi^2$ independentes
    \item Derivação exata via Teorema de Cochran
\end{itemize}

\textbf{3. Síntese Prática (Seção 3)} -- Representação tabular e interpretação:
\begin{itemize}
    \item Tabela ANOVA como resumo operacional
    \item Interpretação dos resultados
\end{itemize}

\subsection{Conexão Lógica entre os Tópicos}

A sequência lógica do relatório segue o padrão clássico de inferência estatística:

$$\text{Modelo} \rightarrow \text{Estimadores} \rightarrow \text{Distribuições} \rightarrow \text{Teste} \rightarrow \text{Aplicação}$$

Esta progressão é \textbf{essencial} porque:
\begin{enumerate}
    \item O modelo e pressupostos garantem que as distribuições exatas $t$ e $F$ sejam válidas
    \item As propriedades dos estimadores (BLUE, normalidade) justificam o uso dos mínimos quadrados
    \item O Teorema de Cochran conecta soma de quadrados com distribuições $\chi^2$ independentes
    \item A decomposição de soma de quadrados aplica o Teorema de Cochran ao teste específico
    \item A estatística $F$ emerge naturalmente como estatística pivotal
\end{enumerate}

\section{Análise Detalhada por Tópico}

\subsection{Seção 1: Modelo e Fundamentos Teóricos}

\subsubsection{1.1 Especificação do Modelo}
\textbf{Conteúdo:} Definição do modelo matricial $\mathbf{y} = \mathbf{X}\beta + \varepsilon$.

\textbf{Importância:} \textcolor{blue}{\textbf{ESSENCIAL}} -- Base para toda a teoria subsequente. Define notação e estrutura.

\textbf{Conexão:} Sem isso, as seções seguintes não têm significado.

\textbf{Recomendação:} \textbf{MANTÉM} (mas pode condensar definições de símbolos em uma linha).

\subsubsection{1.2 Pressupostos Clássicos}
\textbf{Conteúdo:} Definição formal dos 5 pressupostos.

\textbf{Importância:} \textcolor{blue}{\textbf{CRÍTICA}} -- Garante validade das distribuições exatas $F$ e $t$. Sem pressupostos, o teste $F$ não tem distribuição exata.

\textbf{Conexão:} 
\begin{itemize}
    \item Pressuposto de normalidade $\rightarrow$ distribuições $\chi^2$ exatas (Teorema de Cochran)
    \item Não-colinearidade $\rightarrow$ inversibilidade de $\mathbf{X}^T\mathbf{X}$ $\rightarrow$ existência de $\hat{\beta}$
    \item Independência e homocedasticidade $\rightarrow$ estrutura $\sigma^2\mathbf{I}_n$
\end{itemize}

\textbf{Recomendação:} \textbf{MANTÉM} integralmente. É a garantia teórica fundamental.

\subsubsection{1.3 Estimadores de Mínimos Quadrados}
\textbf{Conteúdo (ATUAL):} 
\begin{itemize}
    \item Definição do problema de minimização (linhas 64-67)
    \item Expansão de $S(\beta)$ condensada em uma linha (linha 70)
    \item Derivada e equações normais (linhas 73-76)
    \item Solução $\hat{\beta}$ (linha 80)
    \item Verificação da segunda derivada condensada (linha 83)
\end{itemize}

\textbf{Importância:} \textcolor{blue}{\textbf{ALTA}} -- Necessário para entender de onde vem $\hat{\beta}$ e estabelecer a base para toda a análise posterior.

\textbf{Conexão:}
\begin{itemize}
    \item $\hat{\beta}$ é usado em toda a análise posterior
    \item SSE depende de $\hat{\beta}$ (linha 100)
    \item Propriedades de $\hat{\beta}$ fundamentam o Teorema de Gauss-Markov
\end{itemize}

\textbf{Estado Atual:}
\begin{itemize}
    \item \textbf{Implementado:} Expansão de $S(\beta)$ já condensada em uma linha (linha 70)
    \item \textbf{Implementado:} Verificação da segunda derivada condensada (linha 83)
    \item A seção está otimizada e concisa
\end{itemize}

\textbf{Recomendação:} \textbf{MANTÉM} no estado atual. Já está otimizado conforme as sugestões anteriores.

\subsubsection{1.4 Propriedades do Estimador (Teorema de Gauss-Markov)}
\textbf{Conteúdo (ATUAL):}
\begin{itemize}
    \item Teorema listando 3 propriedades (linhas 85-92)
    \item Demonstração condensada com referência formal (linha 95)
\end{itemize}

\textbf{Importância:} \textcolor{green}{\textbf{MÉDIA-ALTA}} -- Justifica o uso dos mínimos quadrados, mas não é estritamente necessário para derivar o teste $F$.

\textbf{Conexão:}
\begin{itemize}
    \item Propriedade (iii): Normalidade de $\hat{\beta}$ é usada implicitamente
    \item Propriedade (ii): Justifica a escolha do método, mas não é necessária para a distribuição de $F$
    \item Propriedade (i): Não-viesade é usada implicitamente
\end{itemize}

\textbf{Estado Atual:}
\begin{itemize}
    \item \textbf{Implementado:} Demonstração condensada com referência formal (Casella \& Berger, 2002, Teorema 11.2.1)
    \item A prova está mais concisa que a versão anterior
    \item Mantém o rigor teórico necessário
\end{itemize}

\textbf{Recomendação:} \textbf{MANTÉM} no estado atual. A demonstração está condensada adequadamente com referência bibliográfica.

\subsubsection{1.5 Distribuição de $\hat{\sigma}^2$ e Independência}
\textbf{Conteúdo (ATUAL):}
\begin{itemize}
    \item Proposição sobre $\chi^2$ e independência (linhas 103-105)
    \item Demonstração condensada usando projeção ortogonal (linhas 107-109)
\end{itemize}

\textbf{Importância:} \textcolor{blue}{\textbf{CRÍTICA}} -- Este resultado é fundamental para o teste $F$! A independência entre $\hat{\beta}$ e $\hat{\sigma}^2$ é necessária, e a distribuição $\chi^2$ de SSE é usada diretamente na construção de $F$.

\textbf{Conexão:}
\begin{itemize}
    \item SSE aparece no denominador de $F = \frac{SSR/p}{SSE/(n-p-1)}$
    \item A distribuição $\chi^2_{n-p-1}$ de $\frac{SSE}{\sigma^2}$ é essencial
    \item A independência garante que $SSR$ e $SSE$ são independentes (necessário para distribuição $F$)
\end{itemize}

\textbf{Estado Atual:}
\begin{itemize}
    \item \textbf{Implementado:} Demonstração condensada (3 linhas)
    \item \textbf{Implementado:} Referência formal incluída (Seber, Teorema 3.5(iii))
    \item Texto sobre independência simplificado para "ortogonalidade das projeções"
    \item A seção está otimizada mantendo o essencial
\end{itemize}

\textbf{Recomendação:} \textbf{MANTÉM} no estado atual. Proposição essencial com demonstração adequadamente condensada.

\subsubsection{1.6 Teorema de Cochran e Distribuições Qui-Quadrado}
\textbf{Conteúdo (ATUAL):}
\begin{itemize}
    \item Enunciado formal do Teorema de Cochran (linhas 115-117)
    \item Explicação concisa da aplicação (linha 119)
\end{itemize}

\textbf{Importância:} \textcolor{red}{\textbf{MÁXIMA}} -- Este é o \textbf{pilar teórico central} do relatório. Todo o teste $F$ depende deste teorema.

\textbf{Conexão:}
\begin{itemize}
    \item Aplica-se diretamente na Seção 2.3 para decompor $SSR$ e $SSE$
    \item Justifica a independência de $SSR$ e $SSE$
    \item Fundamenta a distribuição $\chi^2$ das somas de quadrados
    \item \textbf{Sem este teorema, não há justificativa teórica para o teste $F$}
\end{itemize}

\textbf{Estado Atual:}
\begin{itemize}
    \item \textbf{Implementado:} Explicação detalhada da aplicação ao modelo de regressão removida
    \item \textbf{Implementado:} Mantida apenas uma frase concisa sobre a importância do teorema
    \item Teorema completo preservado (essencial)
\end{itemize}

\textbf{Recomendação:} \textbf{MANTÉM} no estado atual. Teorema essencial com explicação otimizada.

\subsection{Seção 2: Teste de Hipótese $H_0: \beta_1 = \mathbf{0}_p$}

\subsubsection{2.1 Formulação do Teste}
\textbf{Conteúdo:} Partição de $\beta$ e formulação de $H_0$ (linhas 127-132).

\textbf{Importância:} \textcolor{blue}{\textbf{ESSENCIAL}} -- Define o problema central do relatório.

\textbf{Conexão:} Sem isso, as seções seguintes não têm propósito.

\textbf{Recomendação:} \textbf{MANTÉM} integralmente. Muito conciso já.

\subsubsection{2.2 Estatística F e Distribuição}
\textbf{Conteúdo:}
\begin{itemize}
    \item Definição de $F$ (linhas 137-141)
    \item Teorema da distribuição (linhas 143-148)
\end{itemize}

\textbf{Importância:} \textcolor{blue}{\textbf{ESSENCIAL}} -- Apresenta o resultado principal.

\textbf{Conexão:}
\begin{itemize}
    \item É o objeto de estudo
    \item A Seção 2.3 demonstra este teorema
\end{itemize}

\textbf{Recomendação:} \textbf{MANTÉM}. Apresentação concisa e correta.

\subsubsection{2.3 Derivação via Decomposição de Soma de Quadrados}
\textbf{Conteúdo (ATUAL):}
\begin{itemize}
    \item Modelo reduzido sob $H_0$ condensado (linha 152)
    \item Decomposição em termos de projeções (linhas 154-156)
    \item Aplicação do Teorema de Cochran (linhas 158-161)
    \item Construção da estatística $F$ pivotal sem redundância (linhas 163-168)
\end{itemize}

\textbf{Importância:} \textcolor{red}{\textbf{MÁXIMA}} -- Esta é a \textbf{derivação central} que justifica o teste $F$. É o ponto alto do rigor teórico.

\textbf{Conexão:}
\begin{itemize}
    \item Usa diretamente o Teorema de Cochran (Seção 1.6)
    \item Usa a independência estabelecida na Proposição 1.5
    \item Demonstra o Teorema 2.2 (distribuição de $F$)
    \item É a aplicação prática de toda a teoria desenvolvida
\end{itemize}

\textbf{Estado Atual:}
\begin{itemize}
    \item \textbf{Implementado:} Explicação do modelo reduzido condensada em uma linha (linha 152)
    \item \textbf{Implementado:} Redundância "é uma estatística pivotal" removida (agora aparece apenas uma vez)
    \item \textbf{Implementado:} Texto simplificado entre equações
    \item Estrutura teórica completa mantida
\end{itemize}

\textbf{Recomendação:} \textbf{MANTÉM} no estado atual. Derivação principal otimizada mantendo todo o rigor teórico necessário.

\subsubsection{2.4 Equivalência ao Teste de Razão de Verossimilhança}
\textbf{Estado Atual:} \textcolor{red}{\textbf{ELIMINADA}} -- Esta seção foi completamente removida do relatório.

\textbf{Decisão Implementada:}
\begin{itemize}
    \item Seção 2.4 completamente eliminada conforme recomendação
    \item Economia de 3-4 linhas atingida
    \item Não compromete o essencial teórico do relatório
\end{itemize}

\textbf{Justificativa:}
\begin{itemize}
    \item A equivalência ao LRT adiciona profundidade teórica, mas não é necessária para a validade do teste $F$
    \item O relatório já possui derivação completa e rigorosa via Teorema de Cochran
    \item A eliminação mantém o foco na derivação exata via decomposição de soma de quadrados
\end{itemize}

\textbf{Recomendação:} \textbf{ELIMINAÇÃO MANTIDA}. Decisão correta para otimizar espaço sem perder essência teórica.

\subsubsection{2.5 Região de Rejeição e Testes Complementares}
\textbf{Conteúdo (ATUAL):}
\begin{itemize}
    \item Região de rejeição (linha 172)
    \item Menção a testes $t$ individuais (linha 174)
\end{itemize}

\textbf{Importância:} \textcolor{blue}{\textbf{ESSENCIAL}} -- Aplicação prática do teste.

\textbf{Conexão:} Completa a apresentação do teste.

\textbf{Estado Atual:}
\begin{itemize}
    \item Seção mantida integralmente conforme recomendado
    \item Texto já está conciso e apropriado
\end{itemize}

\textbf{Recomendação:} \textbf{MANTÉM} no estado atual. Seção essencial já otimizada.

\subsection{Seção 3: Análise de Variância e Conclusão}

\subsubsection{3.1 Decomposição ANOVA}
\textbf{Conteúdo (ATUAL):}
\begin{itemize}
    \item Explicação condensada de $SST = SSR + SSE$ (linha 184)
    \item Tabela ANOVA (linhas 186-198)
    \item \textbf{Menção a $R^2$ removida}
\end{itemize}

\textbf{Importância:} \textcolor{blue}{\textbf{ALTA}} -- Representação padrão e prática dos resultados.

\textbf{Conexão:}
\begin{itemize}
    \item Sintetiza os resultados teóricos da Seção 2
    \item É a forma como o teste é aplicado na prática
    \item A tabela é concisa e informativa
\end{itemize}

\textbf{Estado Atual:}
\begin{itemize}
    \item \textbf{Implementado:} Explicação antes da tabela reduzida para 1 linha (linha 184)
    \item \textbf{Implementado:} Removida explicação repetitiva sobre distribuições $\chi^2$
    \item \textbf{Implementado:} Menção a $R^2$ eliminada
    \item Tabela ANOVA preservada integralmente (essencial)
\end{itemize}

\textbf{Recomendação:} \textbf{MANTÉM} no estado atual. Tabela essencial com introdução mínima necessária.

\subsubsection{3.2 Interpretação e Considerações Finais}
\textbf{Conteúdo (ATUAL):} Interpretação condensada e menção a testes $t$ (linha 202).

\textbf{Importância:} \textcolor{orange}{\textbf{MÉDIA}} -- Completa a apresentação com interpretação essencial.

\textbf{Conexão:} Completa a apresentação do teste e conecta com testes complementares.

\textbf{Estado Atual:}
\begin{itemize}
    \item \textbf{Implementado:} Texto condensado para 3 frases pontuais (linha 202)
    \item \textbf{Implementado:} Removida frase redundante sobre verificação de pressupostos
    \item Mantida menção a testes $t$ individuais (necessária)
    \item Interpretação essencial preservada
\end{itemize}

\textbf{Recomendação:} \textbf{MANTÉM} no estado atual. Texto conciso e direto ao ponto.

\section{Estratégia de Otimização para 4 Páginas}

\subsection{Resumo das Economias Propostas}

\begin{table}[h]
\centering
\begin{tabular}{lcc}
\toprule
\textbf{Seção/Tópico} & \textbf{Status} & \textbf{Economia Alcançada} \\
\midrule
1.3 Estimadores MQ & \textcolor{green}{\textbf{IMPLEMENTADO}} & Expansão condensada, verificação reduzida \\
1.4 Gauss-Markov & \textcolor{green}{\textbf{IMPLEMENTADO}} & Demonstração condensada com referência \\
1.5 $\hat{\sigma}^2$ & \textcolor{green}{\textbf{IMPLEMENTADO}} & Demonstração condensada \\
1.6 Cochran & \textcolor{green}{\textbf{IMPLEMENTADO}} & Explicação reduzida \\
2.3 Decomposição SOS & \textcolor{green}{\textbf{IMPLEMENTADO}} & Texto condensado, redundância removida \\
2.4 Equivalência LRT & \textcolor{red}{\textbf{ELIMINADA}} & Seção completamente removida (3-4 linhas) \\
3.1 ANOVA & \textcolor{green}{\textbf{IMPLEMENTADO}} & Explicação reduzida, $R^2$ removido \\
3.2 Considerações & \textcolor{green}{\textbf{IMPLEMENTADO}} & Texto condensado \\
\midrule
\textbf{TOTAL} & \textcolor{blue}{\textbf{OTIMIZADO}} & \textbf{14-19 linhas economizadas} \\
\bottomrule
\end{tabular}
\caption{Status das otimizações implementadas}
\end{table}

\subsection{Priorização por Importância Teórica}

\subsubsection{Prioridade MÁXIMA (Manter Integralmente)}
\begin{enumerate}
    \item Especificação do Modelo (1.1)
    \item Pressupostos Clássicos (1.2)
    \item Teorema de Cochran (1.6) -- \textbf{corpo teórico}
    \item Formulação do Teste (2.1)
    \item Estatística $F$ e Distribuição (2.2)
    \item Derivação via Decomposição SOS (2.3) -- \textbf{derivação principal}
    \item Tabela ANOVA (3.1)
\end{enumerate}

\subsubsection{Prioridade ALTA (Manter com Reduções)}
\begin{enumerate}
    \item Estimadores MQ (1.3) -- reduz expansão algébrica
    \item Distribuição de $\hat{\sigma}^2$ (1.5) -- reduz demonstração
    \item Região de Rejeição (2.5) -- manter
\end{enumerate}

\subsubsection{Prioridade MÉDIA (Condensar ou Eliminar) -- IMPLEMENTADO}
\begin{enumerate}
    \item Teorema de Gauss-Markov (1.4) -- \textcolor{green}{\textbf{CONDENSADO}} com referência formal
    \item Equivalência LRT (2.4) -- \textcolor{red}{\textbf{ELIMINADA}} completamente
    \item Considerações Finais (3.2) -- \textcolor{green}{\textbf{CONDENSADO}} ao mínimo
\end{enumerate}

\subsection{Recomendações Finais}

\subsubsection{Opções Estratégicas}

\textbf{Opção A: Manter Máximo Rigor Teórico}
\begin{itemize}
    \item Mantém todas as demonstrações essenciais
    \item Elimina apenas Seção 2.4 (LRT)
    \item Reduz textos explicativos
    \item \textbf{Resultado esperado:} 4-4.5 páginas
\end{itemize}

\textbf{Opção B: Otimização Balanceada (RECOMENDADA)}
\begin{itemize}
    \item Elimina Seção 2.4 (LRT)
    \item Reduz demonstração de Gauss-Markov a citação
    \item Condensa textos explicativos em todas as seções
    \item Mantém estrutura teórica completa
    \item \textbf{Resultado esperado:} 3.5-4 páginas
\end{itemize}

\textbf{Opção C: Máxima Concisão}
\begin{itemize}
    \item Aplica todas as reduções propostas
    \item Remove $R^2$ e menções secundárias
    \item Foco exclusivo na derivação do teste $F$
    \item \textbf{Resultado esperado:} 3-3.5 páginas
\end{itemize}

\subsubsection{Elementos Não Negociáveis}

Os seguintes elementos \textbf{NÃO PODEM} ser removidos sem comprometer a integridade teórica:

\begin{enumerate}
    \item \textbf{Teorema de Cochran} -- sem ele, não há justificativa teórica para o teste $F$
    \item \textbf{Derivação via Decomposição SOS} -- é a demonstração principal
    \item \textbf{Independência entre $\hat{\beta}$ e $\hat{\sigma}^2$} -- necessária para distribuição $F$
    \item \textbf{Tabela ANOVA} -- representação padrão do teste
    \item \textbf{Pressupostos} -- garantem validade das distribuições exatas
\end{enumerate}

\subsubsection{Redundâncias Identificadas e Resolvidas}

\begin{enumerate}
    \item \textcolor{green}{\textbf{RESOLVIDO:}} Explicação do modelo reduzido condensada na Seção 2.3 (linha 152)
    \item \textcolor{green}{\textbf{RESOLVIDO:}} "Estatística pivotal" mencionada apenas uma vez na Seção 2.3 (redundância removida)
    \item \textcolor{green}{\textbf{RESOLVIDO:}} Explicação antes da tabela ANOVA reduzida (linha 184)
    \item \textcolor{green}{\textbf{MANTIDO:}} Teste $t$ mencionado em 2.5 e 3.2 -- ambas as menções são apropriadas (contextos diferentes)
\end{enumerate}

\section{Conclusão da Análise e Status Final}

O relatório possui uma estrutura teórica sólida e coesa. As otimizações foram implementadas com sucesso:

\begin{enumerate}
    \item \textcolor{green}{\textbf{IMPLEMENTADO:}} Seção 2.4 (Equivalência LRT) eliminada -- economia de 3-4 linhas
    \item \textcolor{green}{\textbf{IMPLEMENTADO:}} Demonstrações secundárias (Gauss-Markov, $\hat{\sigma}^2$) condensadas com referências formais
    \item \textcolor{green}{\textbf{IMPLEMENTADO:}} Textos explicativos condensados mantendo apenas o essencial
    \item \textcolor{blue}{\textbf{MANTIDO:}} Núcleo teórico preservado: Teorema de Cochran + Derivação SOS
    \item \textcolor{blue}{\textbf{MANTIDO:}} Tabela ANOVA preservada como síntese prática
\end{enumerate}

\subsection{Resultado Final}

O relatório atualizado:
\begin{itemize}
    \item \textbf{Mantém o rigor teórico} necessário para nível de doutorado
    \item \textbf{Otimiza o uso do espaço} -- economia estimada de 14-19 linhas
    \item \textbf{Preserva a estrutura teórica completa} com foco na derivação exata via Teorema de Cochran
    \item \textbf{Elimina redundâncias} mantendo coesão e clareza
    \item \textbf{Remove elementos não essenciais} (LRT, $R^2$) sem comprometer a integridade teórica
\end{itemize}

O documento está otimizado para 4 páginas + título, mantendo todos os elementos teóricos essenciais e focando na derivação exata do teste $F$ via Teorema de Cochran e decomposição de soma de quadrados.

\section{Estrutura Final do Documento}

\subsection{Resumo da Organização Atual}

O relatório atualizado possui a seguinte estrutura:

\textbf{Seção 1: Modelo e Fundamentos Teóricos}
\begin{itemize}
    \item 1.1 Especificação do Modelo -- Mantida
    \item 1.2 Pressupostos Clássicos -- Mantida integralmente
    \item 1.3 Estimadores de Mínimos Quadrados -- Condensada
    \item 1.4 Propriedades do Estimador (Gauss-Markov) -- Demonstração condensada com referência
    \item 1.5 Distribuição de $\hat{\sigma}^2$ e Independência -- Demonstração condensada
    \item 1.6 Teorema de Cochran -- Explicação reduzida, teorema completo mantido
\end{itemize}

\textbf{Seção 2: Teste de Hipótese $H_0: \beta_1 = \mathbf{0}_p$}
\begin{itemize}
    \item 2.1 Formulação do Teste -- Mantida
    \item 2.2 Estatística F e Distribuição -- Mantida
    \item 2.3 Derivação via Decomposição SOS -- Texto condensado, estrutura teórica completa
    \item \textcolor{red}{\textbf{2.4 ELIMINADA}} -- Equivalência ao Teste de Razão de Verossimilhança
    \item 2.5 Região de Rejeição e Testes Complementares -- Mantida
\end{itemize}

\textbf{Seção 3: Análise de Variância e Conclusão}
\begin{itemize}
    \item 3.1 Decomposição ANOVA -- Explicação reduzida, $R^2$ removido, tabela mantida
    \item 3.2 Interpretação e Considerações Finais -- Texto condensado
\end{itemize}

\subsection{Melhorias Implementadas}

\begin{enumerate}
    \item \textbf{Concisão}: Todas as seções foram revisadas e condensadas onde apropriado
    \item \textbf{Foco Teórico}: Mantido o rigor matemático necessário para nível de doutorado
    \item \textbf{Eliminação de Redundâncias}: Textos repetitivos removidos
    \item \textbf{Referências Formais}: Demonstrações condensadas incluem referências bibliográficas
    \item \textbf{Estrutura Coesa}: Progressão lógica mantida: Modelo $\rightarrow$ Estimadores $\rightarrow$ Distribuições $\rightarrow$ Teste $\rightarrow$ Aplicação
\end{enumerate}

\subsection{Avaliação Final}

O relatório atualizado:
\begin{itemize}
    \item \textcolor{green}{\textbf{Atende}} ao limite de 4 páginas + título
    \item \textcolor{green}{\textbf{Mantém}} o rigor teórico necessário para doutorado
    \item \textcolor{green}{\textbf{Foca}} na derivação exata via Teorema de Cochran
    \item \textcolor{green}{\textbf{Elimina}} redundâncias e elementos não essenciais
    \item \textcolor{green}{\textbf{Preserva}} todos os elementos teóricos fundamentais
\end{itemize}

O documento está pronto para submissão, apresentando uma fundamentação teórica sólida e completa sobre o teste $H_0: \beta_1 = \mathbf{0}_p$ em regressão normal linear múltipla.

\end{document}

