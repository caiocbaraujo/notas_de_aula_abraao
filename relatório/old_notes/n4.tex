\documentclass{article}
\usepackage[utf8]{inputenc}
\usepackage{amsmath}
\usepackage{amssymb}

\begin{document}

(ii) Para 
\[
A = \tilde{X}^\top \tilde{X} = \{a_{ij}\}_{i,j=1}^{p+1}
\]
\[
\frac{\partial \left[ \tilde{\beta}^\top \tilde{X}^\top \tilde{X} \tilde{\beta} \right]}{\partial \beta_i} = \frac{\partial \left[ \tilde{\beta}^\top A \tilde{\beta} \right]}{\partial \beta_i}
\]
\[
= \frac{\partial \left[ \sum_{l=1}^{p+1} \sum_{m=1}^{p+1} \beta_l \, a_{lm} \, \beta_m \right]}{\partial \beta_i}
\]
\[
= 2\beta_1 a_{1i} + 2\beta_2 a_{2i} + \dots + 2\beta_i a_{ii} + \dots + 2\beta_{p+1} a_{p+1,i}
\]
\[
= 2 \, a_i^\top \beta
\]
en donde \( a_i \) es la \( i \)-ésima columna de \( A \).

Por tanto,
\[
\frac{\partial \left[ \tilde{\beta}^\top \tilde{X}^\top \tilde{X} \tilde{\beta} \right]}{\partial \tilde{\beta}} = 2 A \tilde{\beta} = 2 \tilde{X}^\top \tilde{X} \tilde{\beta}
\]

De ahí, de
\[
\frac{\partial S(\beta)}{\partial \tilde{\beta}} = 0_{(p+1)\times 1}
\]
tenemos que
\[
\tilde{\beta} = \hat{\beta}_{\text{MR}}
\]
\[
\tilde{X}^\top \tilde{y} = \tilde{X}^\top \tilde{X} \hat{\beta}_{\text{MR}}
\]
(ecuaciones normales)

\[
\therefore \quad \hat{\beta}_{\text{MR}} = (\tilde{X}^\top \tilde{X})^{-1} \tilde{X}^\top \tilde{y}
\]

\end{document}