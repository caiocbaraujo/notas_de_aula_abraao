\documentclass{article}
\usepackage[utf8]{inputenc}
\usepackage{amsmath, amssymb}
\usepackage{geometry}
\geometry{a4paper, margin=1in}

\begin{document}

A segunda derivada de $S(\beta)$ é dada por:
\begin{equation}
\frac{\partial^2 S(\beta)}{\partial \beta^2} 
= \frac{\partial \left[ 2 X^T X \beta \right]}{\partial \beta^T} 
= 2 \frac{\partial \left[ A \beta \right]}{\partial \beta^T} 
= 2 A = 2 X^T X
\end{equation}

Note que, $\forall h \in \mathbb{R}^n$, 
\begin{equation}
h^T (2 X^T X) h = 2 \cdot h^T X^T X h = 2 \| X h \|^2
\end{equation}

Para $h \neq 0$, então $2 X^T X$ é positiva definida e $\hat{\beta}_{MQ}$ é o E.M.Q. para $\beta$.

\vspace{0.5cm}
\noindent 24/09/25

\section*{3.3 Critérios de Comparação de Estimadores}

É comum termos vários estimadores para um parâmetro.  
Uma questão importante é como comparar os estimadores.

\subsection*{3.3.1 Viés, variância e erro quadrático médio}

Para as definições dadas a seguir, seja $T(\theta)$ uma função de valor real de $\theta \in \Theta \subset \mathbb{R}^p$.

\textbf{Definição (3.3.1)} Seja $T$ um estimador de valor real para $\theta(\theta)$. O viés de $T$ é dado por:
\begin{equation}
B_\theta(T) = \mathbb{E}_\theta(T) - \gamma(\theta),
\end{equation}
para $\theta \in \Theta$.

\end{document}