\documentclass[12pt,a4paper]{article}
\usepackage[utf8]{inputenc}
\usepackage[T1]{fontenc}
\usepackage[brazil]{babel}
\usepackage{amsmath, amssymb, amsthm}
\usepackage{geometry}
\geometry{margin=2.5cm}
\usepackage{hyperref}
\hypersetup{colorlinks=true,linkcolor=blue,urlcolor=blue}
\usepackage{booktabs}

\title{Sugestões para Melhorias do Relatório\\
\large Elevando o Nível Teórico para Doutorado}
\author{Análise do documento report.tex}
\date{\today}

\begin{document}

\maketitle

\section{Introdução}

Este documento apresenta sugestões detalhadas para elevar o relatório \texttt{report.tex} ao nível teórico esperado em uma dissertação de doutorado. As recomendações focam em rigor matemático, desenvolvimento teórico completo e apresentação adequada dos resultados fundamentais da inferência estatística em regressão linear múltipla.

\section{Sugestões por Seção}

\subsection{Seção: ``Justificativa para o Uso da Estat\'{i}stica F''}

\textbf{Problemas identificados:}
\begin{itemize}
    \item Título inadequado para nível de doutorado: ``Justificativa'' é muito informal
    \item Estrutura em tópicos numerados é superficial para dissertação
    \item Falta derivação matemática rigorosa, especialmente sobre eliminação de $\sigma^2$
    \item Ausência de teoremas e proposições que fundamentem os resultados
\end{itemize}

\textbf{Sugestão de reformulação:}

Renomear para: \textbf{``Fundamentação Teórica da Estat\'{i}stica F''} ou \textbf{``Construção e Propriedades da Estat\'{i}stica F''}

Transformar em texto corrido teórico com as seguintes subseções:

\begin{enumerate}
    \item \textbf{Derivação da Estatística via Teste de Razão de Verossimilhança} - Mostrar que o teste $F$ pode ser derivado como um teste de razão de verossimilhança, conectando com teoria geral de testes de hipótese
    
    \item \textbf{Propriedade de Invariância ao Parâmetro de Nuisance} - Derivação detalhada mostrando como $\sigma^2$ é eliminado:
    
    \begin{quote}
    Partindo de $\frac{SSR}{\sigma^2} \sim \chi^2_p$ e $\frac{SSE}{\sigma^2} \sim \chi^2_{n-p-1}$ independentes, podemos escrever:
    
    $$F = \frac{SSR/p}{SSE/(n-p-1)} = \frac{(SSR/\sigma^2)/p}{(SSE/\sigma^2)/(n-p-1)}$$
    
    Como $\sigma^2$ aparece de forma idêntica no numerador e denominador (ambos divididos por $\sigma^2$), ao formarmos a razão, obtemos uma quantidade que não depende de $\sigma^2$. Mais precisamente, se $U \sim \chi^2_{\nu_1}$ e $V \sim \chi^2_{\nu_2}$ são independentes, então $\frac{U/\nu_1}{V/\nu_2}$ segue uma distribuição $F_{\nu_1, \nu_2}$ que depende apenas dos graus de liberdade, não dos parâmetros de escala. No contexto do teste, $SSR/\sigma^2$ e $SSE/\sigma^2$ têm ambos $\sigma^2$ como fator comum, resultando em uma estatística pivotal.
    \end{quote}
    
    \item \textbf{Teorema Fundamental da Distribuição F} - Formalizar como teorema
    
    \item \textbf{Propriedade de Suficiência e Otimalidade} - Discutir se o teste $F$ possui propriedades ótimas (UMP, invariante, etc.)
\end{enumerate}

\subsection{Derivação Matemática Detalhada da Eliminação de $\sigma^2$}

\textbf{Adicionar após o item sobre eliminação de $\sigma^2$:}

\begin{quote}
\textbf{Proposição:} Sejam $Q_1 \sim \sigma^2 \chi^2_{\nu_1}$ e $Q_2 \sim \sigma^2 \chi^2_{\nu_2}$ variáveis aleatórias independentes com parâmetro de escala comum $\sigma^2$. Então a razão
$$R = \frac{Q_1/\nu_1}{Q_2/\nu_2}$$
segue uma distribuição $F_{\nu_1, \nu_2}$ que não depende de $\sigma^2$.

\textbf{Demonstração:} Como $Q_1 = \sigma^2 U$ e $Q_2 = \sigma^2 V$ onde $U \sim \chi^2_{\nu_1}$ e $V \sim \chi^2_{\nu_2}$ são independentes, temos:
$$R = \frac{(\sigma^2 U)/\nu_1}{(\sigma^2 V)/\nu_2} = \frac{U/\nu_1}{V/\nu_2}$$
que segue $F_{\nu_1, \nu_2}$ independentemente do valor de $\sigma^2$. $\square$
\end{quote}

\subsection{Outras Melhorias Essenciais}

\subsubsection{Seção 1: Modelo e Fundamentos Teóricos}

\textbf{Sugestões:}
\begin{enumerate}
    \item \textbf{Adicionar subseção sobre distribuições assintóticas:} Incluir resultados sobre consistência e distribuição assintótica dos estimadores
    
    \item \textbf{Teorema de Gauss-Markov com demonstração:} O teorema é mencionado mas não demonstrado. Adicionar demonstração completa ou referência detalhada
    
    \item \textbf{Propriedade de independência entre $\hat{\beta}$ e $\hat{\sigma}^2$:} A proposição menciona mas não demonstra. Adicionar demonstração usando propriedades de projeção ortogonal
    
    \item \textbf{Forma matricial completa:} Expandir a especificação do modelo incluindo partição da matriz $\mathbf{X}$:
    $$\mathbf{X} = [\mathbf{1}_n \, | \, \mathbf{X}_1]$$
    onde $\mathbf{1}_n$ é o vetor de uns e $\mathbf{X}_1$ é a matriz $n \times p$ das variáveis explicativas
\end{enumerate}

\subsubsection{Seção 2: Teste de Hipótese}

\textbf{Sugestões:}
\begin{enumerate}
    \item \textbf{Teste de Razão de Verossimilhança:} Derivar o teste $F$ como teste de razão de verossimilhança:
    \begin{quote}
    A função de verossimilhança sob o modelo completo é:
    $$L(\beta, \sigma^2) = (2\pi\sigma^2)^{-n/2} \exp\left\{-\frac{1}{2\sigma^2}(\mathbf{y} - \mathbf{X}\beta)^T(\mathbf{y} - \mathbf{X}\beta)\right\}$$
    
    Sob $H_0: \beta_1 = \mathbf{0}_p$, o estimador de máxima verossimilhança é $\hat{\beta}_0^{(0)} = \bar{y}$. A estatística de razão de verossimilhança é:
    $$\Lambda = \frac{\sup_{H_0} L(\beta, \sigma^2)}{\sup L(\beta, \sigma^2)} = \left(\frac{SSE}{SSE_0}\right)^{n/2}$$
    
    A relação com o teste $F$ é: $-2\log\Lambda = n\log(1 + \frac{p}{n-p-1}F)$.
    \end{quote}
    
    \item \textbf{Derivação via Teoria de Modelos Aninhados:} Conectar com teoria geral de modelos lineares aninhados e teste de significância de blocos de parâmetros
    
    \item \textbf{Distribuição Não-Central:} Sob $H_1$, a estatística $F$ segue distribuição $F$ não-central. Discutir o poder do teste e distribuição não-central
    
    \item \textbf{Derivação da Independência entre $SSR$ e $SSE$:} Usar propriedades de projeção ortogonal. Seja $\mathbf{P} = \mathbf{X}(\mathbf{X}^T\mathbf{X})^{-1}\mathbf{X}^T$ a matriz de projeção no espaço coluna de $\mathbf{X}$ e $\mathbf{P}_0 = \frac{1}{n}\mathbf{1}_n\mathbf{1}_n^T$ a projeção no espaço gerado por $\mathbf{1}_n$. Então:
    $$SSR = \mathbf{y}^T(\mathbf{P} - \mathbf{P}_0)\mathbf{y}, \quad SSE = \mathbf{y}^T(\mathbf{I}_n - \mathbf{P})\mathbf{y}$$
    A independência decorre de que $(\mathbf{P} - \mathbf{P}_0)$ e $(\mathbf{I}_n - \mathbf{P})$ são projeções ortogonais em subespaços ortogonais.
\end{enumerate}

\subsubsection{Seção 3: Análise de Variância}

\textbf{Sugestões:}
\begin{enumerate}
    \item \textbf{Decomposição Ortogonal:} Mostrar que a decomposição $SST = SSR + SSE$ é uma decomposição ortogonal usando teoria de projeção:
    $$\mathbf{y} - \bar{y}\mathbf{1}_n = (\hat{\mathbf{y}} - \bar{y}\mathbf{1}_n) + (\mathbf{y} - \hat{\mathbf{y}})$$
    onde os vetores são ortogonais
    
    \item \textbf{Teorema de Cochran:} Conectar com o Teorema de Cochran sobre distribuição qui-quadrado de formas quadráticas
    
    \item \textbf{$R^2$ Ajustado e outras medidas:} Incluir discussão sobre $R^2$ ajustado, critérios de informação (AIC, BIC), e suas relações com o teste $F$
    
    \item \textbf{Interpretação Geométrica:} Adicionar interpretação geométrica usando espaços vetoriais e projeções ortogonais
\end{enumerate}

\section{Melhorias Estruturais e de Apresentação}

\subsection{Formatação e Rigor}

\begin{enumerate}
    \item \textbf{Ambientes de Teoremas:} Todos os resultados principais devem estar em ambientes de teorema/proposição/lema com demonstrações completas ou referências explícitas
    
    \item \textbf{Notação Consistente:} Garantir que toda notação seja definida explicitamente antes do uso
    
    \item \textbf{Referências Teóricas:} Cada teorema/proposição deve ter referência bibliográfica explícita (ex: ``Teorema 3.2.1 de Casella \& Berger, 2002'')
    
    \item \textbf{Índice de Símbolos:} Considerar adicionar uma tabela de notação para facilitar leitura
    
    \item \textbf{Lema Preparatórios:} Adicionar lemas auxiliares quando necessário (ex: lema sobre distribuição de formas quadráticas)
\end{enumerate}

\subsection{Conteúdo Teórico Adicional}

\begin{enumerate}
    \item \textbf{Propriedades do Teste:}
    \begin{itemize}
        \item Consistência do teste $F$
        \item Poder do teste e distribuição não-central
        \item Invariância do teste
        \item Relação com teste de Wald e teste de escore
    \end{itemize}
    
    \item \textbf{Conexões com Outras Teorias:}
    \begin{itemize}
        \item Relação com teoria geral de testes de hipótese
        \item Conexão com análise de variância (ANOVA)
        \item Relação com teoria de modelos lineares generalizados
        \item Conexão com inferência bayesiana (opcional, mas enriquecedor)
    \end{itemize}
    
    \item \textbf{Condições de Regularidade:}
    \begin{itemize}
        \item Discutir quando os pressupostos podem ser relaxados
        \item Resultados assintóticos quando normalidade não vale
        \item Efeitos de violação de pressupostos
    \end{itemize}
\end{enumerate}

\section{Exemplo de Reformulação: Seção sobre Estat\'{i}stica F}

\textbf{Texto atual (inadequado para doutorado):}
\begin{quote}
\subsection{Justificativa para o Uso da Estat\'{i}stica F}

A estatística $F$ é utilizada por razões teóricas fundamentais:

\begin{enumerate}
    \item Natureza multivariada do teste...
    \item Estrutura distribucional...
    \item Eliminação de $\sigma^2$ desconhecido...
    \item Interpretação como comparação de variâncias...
\end{enumerate}
\end{quote}

\textbf{Texto sugerido (nível doutorado):}
\begin{quote}
\subsection{Fundamentação Teórica da Estat\'{i}stica F}

A escolha da estatística $F$ para o teste $H_0: \beta_1 = \mathbf{0}_p$ decorre de propriedades teóricas fundamentais que conectam a teoria de modelos lineares com a teoria geral de testes de hipótese. Nesta seção, desenvolvemos a fundamentação completa, mostrando como a estrutura distribucional do modelo normal linear conduz naturalmente à estatística $F$ como uma estatística pivotal apropriada.

\subsubsection{Construção como Estatística Pivotal}

Sob os pressupostos do modelo normal linear, temos que $\frac{SSR}{\sigma^2} \sim \chi^2_p$ e $\frac{SSE}{\sigma^2} \sim \chi^2_{n-p-1}$ são independentes, conforme estabelecido no Teorema 2.3. Segue imediatamente que a razão dessas quantidades, quando adequadamente normalizadas por seus graus de liberdade, elimina o parâmetro de nuisance $\sigma^2$.

\begin{proposicao}[Eliminação do Parâmetro de Nuisance]
Sejam $Q_1 \sim \sigma^2 \chi^2_{\nu_1}$ e $Q_2 \sim \sigma^2 \chi^2_{\nu_2}$ variáveis aleatórias independentes com parâmetro de escala comum $\sigma^2 > 0$. Então a estatística
$$R = \frac{Q_1/\nu_1}{Q_2/\nu_2}$$
segue uma distribuição $F_{\nu_1, \nu_2}$ que não depende de $\sigma^2$.
\end{proposicao}

\begin{proof}
Como $Q_1 = \sigma^2 U$ e $Q_2 = \sigma^2 V$ onde $U \sim \chi^2_{\nu_1}$ e $V \sim \chi^2_{\nu_2}$ são independentes, temos:
$$R = \frac{(\sigma^2 U)/\nu_1}{(\sigma^2 V)/\nu_2} = \frac{\sigma^2}{\sigma^2} \cdot \frac{U/\nu_1}{V/\nu_2} = \frac{U/\nu_1}{V/\nu_2}$$
A última quantidade segue uma distribuição $F_{\nu_1, \nu_2}$ por definição da distribuição $F$, e não depende de $\sigma^2$.
\end{proof}

Aplicando este resultado ao contexto do teste, obtemos que $F = \frac{MSR}{MSE}$ é uma estatística pivotal, cuja distribuição sob $H_0$ é completamente especificada e não depende de parâmetros desconhecidos.

\subsubsection{Derivação via Teste de Razão de Verossimilhança}

O teste $F$ pode ser derivado como uma transformação monotônica da estatística de razão de verossimilhança. Sob $H_0: \beta_1 = \mathbf{0}_p$, o modelo reduzido possui estimadores de máxima verossimilhança $\hat{\beta}_0^{(0)} = \bar{y}$ e $\hat{\sigma}^2_0 = \frac{1}{n}\sum_{i=1}^n (y_i - \bar{y})^2$. Sob o modelo completo, temos $\hat{\beta} = (\mathbf{X}^T\mathbf{X})^{-1}\mathbf{X}^T\mathbf{y}$ e $\hat{\sigma}^2 = \frac{SSE}{n}$.

A estatística de razão de verossimilhança é:
$$\Lambda = \frac{\sup_{H_0} L(\beta, \sigma^2)}{\sup L(\beta, \sigma^2)} = \left(\frac{SSE}{SSE_0}\right)^{n/2} = \left(\frac{n-p-1}{n-p-1 + pF}\right)^{n/2}$$

onde utilizamos a relação $SSE_0 = SSE + SSR$ e $SSR = p \cdot MSR = p \cdot F \cdot MSE = \frac{p(n-p-1)}{n-p-1}F \cdot MSE$. Esta conexão estabelece o teste $F$ como parte da teoria geral de testes de hipótese...

\subsubsection{Propriedades de Otimalidade}

[Desenvolver discussão sobre propriedades do teste, relação com teoria de testes UMP, etc.]
\end{quote}

\section{Resumo das Prioridades}

\textbf{Prioridade Alta:}
\begin{enumerate}
    \item Reformular seção ``Justificativa para o Uso da Estat\'{i}stica F'' em texto teórico corrido com derivações
    \item Adicionar demonstração completa da eliminação de $\sigma^2$
    \item Incluir derivação via teste de razão de verossimilhança
    \item Adicionar demonstração da independência entre $\hat{\beta}$ e $\hat{\sigma}^2$
\end{enumerate}

\textbf{Prioridade Média:}
\begin{enumerate}
    \item Conectar com Teorema de Cochran
    \item Adicionar discussão sobre distribuição não-central e poder
    \item Incluir interpretação geométrica
    \item Expandir referências bibliográficas
\end{enumerate}

\textbf{Prioridade Baixa (mas enriquecedor):}
\begin{enumerate}
    \item Resultados assintóticos
    \item Conexões com outros testes (Wald, escore)
    \item Tabela de notação
    \item Apêndices com demonstrações auxiliares
\end{enumerate}

\end{document}

