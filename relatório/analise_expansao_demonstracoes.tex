\documentclass[12pt,a4paper]{article}
\usepackage[utf8]{inputenc}
\usepackage[T1]{fontenc}
\usepackage[brazil]{babel}
\usepackage{amsmath, amssymb, amsthm}
\usepackage{geometry}
\geometry{margin=2.5cm}
\usepackage{booktabs}
\usepackage{enumitem}
\usepackage{xcolor}
\usepackage{hyperref}
\hypersetup{colorlinks=true,linkcolor=blue,urlcolor=blue}

% Definições de teoremas
\theoremstyle{definition}
\newtheorem{definicao}{Definição}[section]
\theoremstyle{plain}
\newtheorem{teorema}{Teorema}[section]
\newtheorem{proposicao}{Proposição}[section]

\title{Análise de Expansão do Relatório\\
\large Identificação de Seções para Ampliação com Demonstrações\\
\large Objetivo: Completar 5 Páginas sem Exceder}
\author{Análise Estrutural}
\date{\today}

\begin{document}

\maketitle

\section{Objetivo da Análise}

Este documento analisa o arquivo \texttt{report.tex} para identificar quais seções podem ser ampliadas com demonstrações matemáticas e desenvolvimentos teóricos adicionais, com o objetivo de preencher adequadamente o espaço disponível na quinta página, mantendo o limite de 5 páginas estabelecido pelo professor.

\section{Análise da Estrutura Atual}

\subsection{Distribuição de Conteúdo por Seção}

O relatório atual (\texttt{report.tex}) está organizado da seguinte forma:

\begin{enumerate}
    \item \textbf{Página 1:} Título e informações do documento
    \item \textbf{Páginas 2-4:} Conteúdo principal (seções teóricas)
    \item \textbf{Página 5:} Análise de Variância, Conclusão e Referências (com espaço disponível)
\end{enumerate}

\subsection{Espaço Disponível na Última Página}

A última página (Página 5) contém:
\begin{itemize}
    \item Seção 4: Análise de Variância e Conclusão (parcial)
    \item Tabela ANOVA
    \item Subseção de Interpretação (breve)
    \item Referências Bibliográficas (3 itens)
    \item Espaço estimado disponível: aproximadamente 30-40\% da página
\end{itemize}

\section{Seções Identificadas para Ampliação}

\subsection{Prioridade ALTA: Ampliações Recomendadas}

\subsubsection{1. Seção 3.1: Propriedades do Estimador (Página 2)}

\textbf{Estado Atual:}
\begin{itemize}
    \item Teorema enunciado com 3 propriedades
    \item Prova breve e concisa (3 linhas)
    \item Menciona Teorema de Gauss-Markov sem detalhamento
\end{itemize}

\textbf{Ampliação Proposta:}
\begin{enumerate}
    \item \textbf{Adicionar demonstração detalhada de $E[\hat{\beta}] = \beta$:}
    \begin{align*}
    E[\hat{\beta}] &= E[(\mathbf{X}^T\mathbf{X})^{-1}\mathbf{X}^T\mathbf{y}] \\
    &= (\mathbf{X}^T\mathbf{X})^{-1}\mathbf{X}^T E[\mathbf{y}] \\
    &= (\mathbf{X}^T\mathbf{X})^{-1}\mathbf{X}^T\mathbf{X}\beta \\
    &= \beta
    \end{align*}
    
    \item \textbf{Expandir explicação do Teorema de Gauss-Markov:} Adicionar 2-3 linhas explicando por que o estimador de mínimos quadrados é ótimo na classe de estimadores lineares não-viesados, mencionando que $\mathrm{Var}(\tilde{\beta}) = \mathrm{Var}(\hat{\beta}) + \sigma^2\mathbf{C}\mathbf{C}^T$ implica que a diferença de variâncias é semidefinida positiva.
    
    \item \textbf{Demonstração da distribuição normal de $\hat{\beta}$:} Como $\hat{\beta} = (\mathbf{X}^T\mathbf{X})^{-1}\mathbf{X}^T\mathbf{y}$ é uma transformação linear de $\mathbf{y} \sim N(\mathbf{X}\beta, \sigma^2\mathbf{I}_n)$, segue que $\hat{\beta} \sim N(\beta, \sigma^2(\mathbf{X}^T\mathbf{X})^{-1})$ por propriedades de transformações lineares de vetores normais.
\end{enumerate}

\textbf{Espaço a ser adicionado:} 8-10 linhas (aproximadamente meio parágrafo)

\textbf{Justificativa:} Amplia o rigor teórico sem excesso, completando demonstrações essenciais.

\subsubsection{2. Seção 3.2: Distribuição de $\hat{\sigma}^2$ e Independência (Página 2)}

\textbf{Estado Atual:}
\begin{itemize}
    \item Proposição enunciada
    \item Prova menciona Teorema de Cochran mas não detalha a aplicação
    \item Menciona ortogonalidade sem demonstrar explicitamente
\end{itemize}

\textbf{Ampliação Proposta:}
\begin{enumerate}
    \item \textbf{Demonstrar explicitamente a ortogonalidade das projeções:}
    \begin{itemize}
        \item Mostrar que $\mathbf{P}(\mathbf{I}_n - \mathbf{P}) = \mathbf{0}$
        \item Explicar que isso garante independência via Teorema de Cochran
    \end{itemize}
    
    \item \textbf{Detalhar aplicação do Teorema de Cochran:}
    \begin{itemize}
        \item Mostrar que $\mathbf{P}$ e $\mathbf{I}_n - \mathbf{P}$ são idempotentes e ortogonais
        \item Verificar que $\mathrm{tr}(\mathbf{I}_n - \mathbf{P}) = n - \mathrm{tr}(\mathbf{P}) = n - (p+1) = n-p-1$
        \item Concluir que $\frac{SSE}{\sigma^2} = \frac{\mathbf{y}^T(\mathbf{I}_n - \mathbf{P})\mathbf{y}}{\sigma^2} \sim \chi^2_{n-p-1}$
    \end{itemize}
    
    \item \textbf{Demonstrar independência entre $\hat{\beta}$ e $\hat{\sigma}^2$:} Como $\hat{\beta} = (\mathbf{X}^T\mathbf{X})^{-1}\mathbf{X}^T\mathbf{y}$ depende de $\mathbf{P}\mathbf{y}$ e $SSE = \mathbf{y}^T(\mathbf{I}_n - \mathbf{P})\mathbf{y}$ depende de $(\mathbf{I}_n - \mathbf{P})\mathbf{y}$, e como $\mathbf{P}(\mathbf{I}_n - \mathbf{P}) = \mathbf{0}$, segue que $\mathbf{P}\mathbf{y}$ e $(\mathbf{I}_n - \mathbf{P})\mathbf{y}$ são independentes, logo $\hat{\beta}$ e $\hat{\sigma}^2$ são independentes.
\end{enumerate}

\textbf{Espaço a ser adicionado:} 12-15 linhas (aproximadamente um parágrafo completo)

\textbf{Justificativa:} Aplicação do Teorema de Cochran é fundamental e merece desenvolvimento detalhado.

\subsubsection{3. Seção 3.4: Teorema de Cochran (Página 3)}

\textbf{Estado Atual:}
\begin{itemize}
    \item Teorema apenas enunciado
    \item Sem demonstração ou explicação intuitiva
    \item Uma frase sobre importância (muito breve)
\end{itemize}

\textbf{Ampliação Proposta:}
\begin{enumerate}
    \item \textbf{Adicionar explicação intuitiva antes do enunciado:}
    \begin{itemize}
        \item Explicar que o teorema estabelece condições sob as quais formas quadráticas de vetores normais seguem distribuições qui-quadrado independentes
        \item Mencionar que isso é essencial para a validade exata do teste F
        \item Explicar a intuição: decomposição ortogonal do espaço em subespaços independentes
    \end{itemize}
    
    \item \textbf{Adicionar nota sobre aplicação:} Explicar brevemente (3-4 linhas) como o teorema será aplicado na Seção 4 para fundamentar a decomposição de soma de quadrados no teste $H_0: \beta_1 = \mathbf{0}_p$.
    
    \item \textbf{Opção de demonstração conceitual:} Se o espaço permitir, adicionar esboço de demonstração (não completa, mas conceitual) explicando os passos principais:
    \begin{itemize}
        \item Ortogonalidade das projeções implica independência das formas quadráticas
        \item Idempotência garante distribuições qui-quadrado
        \item Rank determina graus de liberdade
    \end{itemize}
\end{enumerate}

\textbf{Espaço a ser adicionado:} 10-12 linhas (aproximadamente três quartos de parágrafo)

\textbf{Justificativa:} O Teorema de Cochran é fundamental e merece maior desenvolvimento, facilitando compreensão da aplicação posterior.

\subsubsection{4. Seção 4.2: Derivação via Decomposição (Página 3)}

\textbf{Estado Atual:}
\begin{itemize}
    \item Explicação da decomposição de soma de quadrados
    \item Menciona aplicação do Teorema de Cochran mas não detalha verificações
    \item Conclusão sobre distribuição F apresentada diretamente
\end{itemize}

\textbf{Ampliação Proposta:}
\begin{enumerate}
    \item \textbf{Verificar explicitamente condições do Teorema de Cochran:}
    \begin{itemize}
        \item Mostrar que $\mathbf{Q}_1 = \mathbf{P} - \mathbf{P}_0$ é idempotente: $(\mathbf{P} - \mathbf{P}_0)^2 = \mathbf{P}^2 - \mathbf{P}\mathbf{P}_0 - \mathbf{P}_0\mathbf{P} + \mathbf{P}_0^2 = \mathbf{P} - \mathbf{P}_0 - \mathbf{P}_0 + \mathbf{P}_0 = \mathbf{P} - \mathbf{P}_0$ (pois $\mathbf{P}\mathbf{P}_0 = \mathbf{P}_0$ e $\mathbf{P}_0\mathbf{P} = \mathbf{P}_0$ quando $\mathbf{P}_0$ projeta em subespaço de $\mathbf{P}$)
        \item Verificar que $\mathbf{Q}_2 = \mathbf{I}_n - \mathbf{P}$ é idempotente (já estabelecido)
        \item Verificar ortogonalidade: $\mathbf{Q}_1\mathbf{Q}_2 = (\mathbf{P} - \mathbf{P}_0)(\mathbf{I}_n - \mathbf{P}) = \mathbf{P} - \mathbf{P}^2 - \mathbf{P}_0 + \mathbf{P}_0\mathbf{P} = \mathbf{0}$
        \item Mostrar que $\mathrm{tr}(\mathbf{Q}_1) = \mathrm{tr}(\mathbf{P}) - \mathrm{tr}(\mathbf{P}_0) = (p+1) - 1 = p$
    \end{itemize}
    
    \item \textbf{Demonstrar que sob $H_0$ temos $\mathbf{Q}_1\mathbf{X}\beta = \mathbf{0}$:} Quando $\beta_1 = \mathbf{0}_p$, temos $\beta = (\beta_0, \mathbf{0}_p^T)^T$ e $\mathbf{X}\beta = \beta_0\mathbf{1}_n$. Como $(\mathbf{P} - \mathbf{P}_0)\mathbf{1}_n = \mathbf{1}_n - \mathbf{1}_n = \mathbf{0}$ quando $\mathbf{P}_0$ projeta no espaço gerado por $\mathbf{1}_n$ que está contido no espaço de $\mathbf{P}$, segue que $\mathbf{Q}_1\mathbf{X}\beta = \mathbf{0}$.
    
    \item \textbf{Explicar construção da estatística pivotal:} Detalhar (2-3 linhas) por que $F$ é pivotal, explicando que elimina $\sigma^2$ ao dividir duas variáveis qui-quadrado independentes por seus graus de liberdade.
\end{enumerate}

\textbf{Espaço a ser adicionado:} 15-18 linhas (aproximadamente um parágrafo e meio)

\textbf{Justificativa:} Esta seção é o cerne teórico do trabalho e merece desenvolvimento completo das verificações.

\subsection{Prioridade MÉDIA: Ampliações Opcionais}

\subsubsection{5. Seção 4.1: Nota sobre Equivalência ao LRT (Página 3)}

\textbf{Estado Atual:}
\begin{itemize}
    \item Menciona equivalência ao Teste de Razão de Verossimilhança
    \item Não demonstra a equivalência
    \item Menciona que F é uniformemente mais poderoso mas não desenvolve
\end{itemize}

\textbf{Ampliação Proposta (se espaço permitir):}
\begin{enumerate}
    \item \textbf{Desenvolver brevemente a equivalência:}
    \begin{itemize}
        \item Mostrar que a estatística LRT para $H_0: \beta_1 = \mathbf{0}_p$ é $-2\log\Lambda = n\log(SSE_0/SSE)$
        \item Mostrar que esta é monotonicamente relacionada a $F$ através de $F = \frac{n-p-1}{p}\left[\exp\left(\frac{-2\log\Lambda}{n}\right) - 1\right]$
        \item Concluir que como são monotonicamente relacionadas, os testes são equivalentes
    \end{itemize}
\end{enumerate}

\textbf{Espaço a ser adicionado:} 6-8 linhas (opcional, depende do espaço disponível)

\textbf{Justificativa:} Enriquece o trabalho, mas não é estritamente necessário. Adicionar apenas se houver espaço após as ampliações de prioridade ALTA.

\subsubsection{6. Seção 4.3: Testes Complementares (Página 3)}

\textbf{Estado Atual:}
\begin{itemize}
    \item Menciona testes $t$ individuais brevemente
    \item Não desenvolve a relação entre teste F global e testes t
\end{itemize}

\textbf{Ampliação Proposta (se espaço permitir):}
\begin{enumerate}
    \item \textbf{Explicar relação entre teste F e testes t:}
    \begin{itemize}
        \item Mencionar que quando $p = 1$, o teste F é equivalente ao quadrado do teste t
        \item Explicar que o teste F global é necessário quando há múltiplas variáveis explicativas
        \item Mencionar que testes t individuais são condicionais à inclusão de outras variáveis no modelo
    \end{itemize}
\end{enumerate}

\textbf{Espaço a ser adicionado:} 4-5 linhas (muito opcional)

\textbf{Justificativa:} Útil para contexto, mas menos prioritário que as ampliações de rigor teórico.

\subsubsection{7. Seção 4.4: Interpretação Estendida (Página 5)}

\textbf{Estado Atual:}
\begin{itemize}
    \item Interpretação muito concisa (2 parágrafos)
    \item Espaço disponível na página 5
\end{itemize}

\textbf{Ampliação Proposta:}
\begin{enumerate}
    \item \textbf{Adicionar interpretação da estatística F:}
    \begin{itemize}
        \item Explicar que F mede a razão entre variância explicada e variância residual
        \item Valores grandes indicam que MSR é significativamente maior que MSE
        \item Isso sugere que as variáveis explicativas reduzem substancialmente a variabilidade não explicada
    \end{itemize}
    
    \item \textbf{Adicionar nota sobre poder do teste:}
    \begin{itemize}
        \item Mencionar que o poder do teste F aumenta com o tamanho da amostra
        \item Mencionar que o poder depende da distância de $\beta_1$ de $\mathbf{0}_p$ sob $H_1$
        \item Explicar que rejeitar $H_0$ não implica que todas as variáveis são importantes, apenas que pelo menos uma é
    \end{itemize}
    
    \item \textbf{Adicionar considerações práticas:}
    \begin{itemize}
        \item Mencionar importância de verificar pressupostos antes de aplicar o teste
        \item Sugerir análise de resíduos para validar pressupostos
        \item Mencionar que violações podem afetar a validade do teste
    \end{itemize}
\end{enumerate}

\textbf{Espaço a ser adicionado:} 8-12 linhas (ideal para preencher espaço da página 5)

\textbf{Justificativa:} Preenche naturalmente o espaço disponível na última página, enriquecendo a conclusão sem comprometer o limite de páginas.

\section{Plano de Ação Recomendado}

\subsection{Estratégia de Implementação}

\textbf{Etapa 1: Ampliações de Prioridade ALTA (obrigatórias)}
\begin{enumerate}
    \item Expandir demonstração de propriedades do estimador (Seção 3.1): +8-10 linhas
    \item Detalhar demonstração de distribuição de $\hat{\sigma}^2$ (Seção 3.2): +12-15 linhas
    \item Ampliar explicação do Teorema de Cochran (Seção 3.4): +10-12 linhas
    \item Detalhar derivação via decomposição (Seção 4.2): +15-18 linhas
\end{enumerate}

\textbf{Total estimado:} 45-55 linhas adicionais

\textbf{Etapa 2: Avaliação de Espaço Restante}
\begin{itemize}
    \item Após implementar ampliações de prioridade ALTA, avaliar espaço disponível na página 5
    \item Se espaço permitir, implementar ampliações de prioridade MÉDIA
\end{itemize}

\textbf{Etapa 3: Ajustes Finais}
\begin{itemize}
    \item Expandir interpretação na Seção 4.4 para preencher espaço restante na página 5
    \item Ajustar espaçamentos e formatação conforme necessário
    \item Garantir que o documento não exceda 5 páginas
\end{itemize}

\subsection{Distribuição Estimada de Espaço}

Após implementação das ampliações recomendadas:

\begin{itemize}
    \item \textbf{Páginas 1-2:} Conteúdo atual + ampliações das Seções 3.1 e 3.2
    \item \textbf{Página 3:} Conteúdo atual + ampliações das Seções 3.4 e 4.2 (parcial)
    \item \textbf{Página 4:} Conteúdo atual + continuação da Seção 4.2
    \item \textbf{Página 5:} Análise de Variância + Interpretação ampliada (preenchendo espaço disponível)
\end{itemize}

\section{Recomendações Finais}

\subsection{Ordem de Prioridade}

\begin{enumerate}
    \item \textcolor{red}{\textbf{PRIORIDADE MÁXIMA:}} Ampliar Seção 4.2 (Derivação via Decomposição) -- cerne teórico do trabalho
    \item \textcolor{red}{\textbf{PRIORIDADE ALTA:}} Detalhar Seção 3.2 (Distribuição de $\hat{\sigma}^2$) -- aplicação fundamental do Teorema de Cochran
    \item \textcolor{orange}{\textbf{PRIORIDADE MÉDIA-ALTA:}} Expandir Seção 3.4 (Teorema de Cochran) -- fundamentação teórica essencial
    \item \textcolor{orange}{\textbf{PRIORIDADE MÉDIA:}} Ampliar Seção 3.1 (Propriedades do Estimador) -- completude de demonstrações
    \item \textcolor{blue}{\textbf{PRIORIDADE BAIXA:}} Expandir Seção 4.4 (Interpretação) -- preencher espaço da página 5
    \item \textcolor{blue}{\textbf{OPCIONAL:}} Adicionar nota sobre equivalência LRT e testes complementares -- apenas se espaço permitir
\end{enumerate}

\subsection{Considerações Importantes}

\begin{enumerate}
    \item \textbf{Manter rigor matemático:} Todas as ampliações devem manter o nível de rigor apropriado para nível de doutorado
    \item \textbf{Coesão com objetivo:} Todas as demonstrações devem contribuir para a fundamentação teórica do teste $H_0: \beta_1 = \mathbf{0}_p$
    \item \textbf{Equilíbrio:} Não exceder o limite de 5 páginas, mas preencher adequadamente o espaço disponível
    \item \textbf{Clareza:} Demonstrações devem ser claras e bem estruturadas, facilitando compreensão sem comprometer rigor
\end{enumerate}

\section{Conclusão}

As ampliações propostas focam em:
\begin{itemize}
    \item \textbf{Rigor teórico:} Completar demonstrações essenciais para nível de doutorado
    \item \textbf{Fundamentação:} Desenvolver detalhadamente a aplicação do Teorema de Cochran
    \item \textbf{Completude:} Garantir que todas as afirmações principais tenham justificativa teórica adequada
    \item \textbf{Otimização de espaço:} Preencher adequadamente as 5 páginas sem exceder
\end{itemize}

A implementação gradual (prioridade ALTA primeiro, depois avaliar espaço para prioridade MÉDIA) garante que o documento atinja o objetivo de completude teórica mantendo o limite de páginas estabelecido.

\end{document}

