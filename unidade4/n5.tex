\[
\psi_Y(x) =
\begin{cases}
1, & x \in R_{c_i}, \\
0, & x \in R_{c_i}^c
\end{cases}
\]

b) $Y$ aleatorizado: O teste é definido pela função crítica:

\[
\psi_Y(x) =
\begin{cases}
1, & x \in R_{c_i}, \\
\delta, & x \in R_{\delta_i}, \\
0, & x \in (R_{c_i} \cup R_{\delta_i})^c
\end{cases}
\]

\textbf{Exemplo 4:} Sejam $X_1, \ldots, X_n$ i.i.d. $X \sim N(\theta, 25)$. Neste caso, $\alpha = 1.2^n$. Considere o teste $Y$: Rejeitar $H_0: \theta \leq 17$ se $\alpha$, i.e.,

\[
\bar{X}_n > 17 + \frac{5}{\sqrt{n}}
\]

$Y$ não aleatorizado com função crítica:

\[
\psi(x) =
\begin{cases}
1, & x \in \{ x \in \mathbb{R}^n : \bar{x}_n > 17 + \frac{5}{\sqrt{n}} \}, \\
0, & \text{c.c.}
\end{cases}
\]

\textbf{Exemplo 5:} Sejam $X_1, \ldots, X_n$ uma a.a. de $X_i \sim \text{Bernoulli}(p)$. Neste caso $\alpha = \sum_{i=1}^n X_i \in \{0, \ldots, n\}$. Considere um teste $Y$ para ...
