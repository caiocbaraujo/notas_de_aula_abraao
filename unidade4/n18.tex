Adotamos a hipótese alternativa como $H_1: \sigma = \sigma_1 (< \sigma_0)$. Assim, o LNP se aplica. A função de verossimilhança é dada por:

\begin{equation}
\frac{L_1}{L_0} = L(\sigma_i, \alpha) = \prod_{i=1}^n f(\alpha_i, \sigma_k)_{N(0, \sigma_k^2)} 
= (2\pi \sigma_k^2)^{-n/2} \exp\left\{ -\frac{1}{2\sigma_k^2} \sum_{i=1}^n \alpha_i^2 \right\}
\end{equation}

Portanto, $H_0$ é rejeitada quando a razão é maior que $k$ (para $\alpha \in \mathbb{R}^n$):

\begin{equation}
\mathcal{R}_c = \left\{ \alpha \in \mathcal{X} : \frac{L_1}{L_0} > k \right\}
\end{equation}

\begin{equation}
= \left\{ \alpha \in \mathcal{X} : \left( \frac{\sigma_1}{\sigma_0} \right)^{-n/2} \exp\left[ -\frac{1}{2} \sum_{i=1}^n \alpha_i^2 \left( \frac{1}{\sigma_1^2} - \frac{1}{\sigma_0^2} \right) \right] > k \right\}
\end{equation}

\begin{equation}
= \left\{ \alpha \in \mathcal{X} : -\frac{1}{2} \sum_{i=1}^n \alpha_i^2 \left( \frac{\sigma_0^2 - \sigma_1^2}{\sigma_0^2 \sigma_1^2} \right) > \log\left[ k \left( \frac{\sigma_1}{\sigma_0} \right)^{n/2} \right] \right\}
\end{equation}

\begin{equation}
= \left\{ \alpha \in \mathcal{X} : \sum_{i=1}^n \alpha_i^2 \frac{1}{\sigma_0^2} < \frac{2\sigma_0^2 \sigma_1^2}{\sigma_0^2 - \sigma_1^2} \log\left[ k \left( \frac{\sigma_1}{\sigma_0} \right)^{n/2} \right] \right\}
\end{equation}

\begin{equation}
= \left\{ \alpha \in \mathcal{X} : \sum_{i=1}^n \left( \frac{\alpha_i}{\sigma_0} \right)^2 < k_1 \right\}
\end{equation}

Definimos a função $Q: \mathcal{X} \to \mathbb{R}^+$ tal que...
