\section*{Teste UMP via Lema de Neyman Pearson}

Inicialmente, devemos fixar um valor arbitrário $\theta_1 \in \Theta$ tal que $\theta_1 > \theta_0$. A hipótese alternativa em (a) pode ser reescrita como
\begin{equation}
H_1: \theta = \theta_1.
\end{equation}

Agora, temos um problema de duas hipóteses simples e, pelo LNP, existe um teste NP para
\begin{equation}
H_0: \theta = \theta_0 \quad \times \quad H_1: \theta = \theta_1.
\end{equation}

Se este teste particular não é afetado pela escolha de um valor para $\theta_1$, então dizemos que ele é UMP.

\subsection*{Exemplo 9}
Sejam $X_1, \ldots, X_n$ uma a.a. de $N(\theta, \sigma^2)$, em que $\sigma \in \mathbb{R}^+$ é desconhecido. Fixando o nível $\alpha \in (0,1)$, considere que se deseja obter o teste UMP de nível $\alpha$ para
\begin{equation}
H_0: \sigma = \sigma_0 \quad \times \quad H_1: \sigma < \sigma_0,
\end{equation}
em que $\sigma_0 > 0$.

\textbf{Solução:} Fixamos $\sigma_1 \in \Theta = \mathbb{R}^+_*$ tal que $\sigma_1 < \sigma_0$ e então...
