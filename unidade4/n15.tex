Um que
\begin{equation}
P_{H_0} \left( \sum_{i=1}^n X_i = k_1 \right) = e^{-n\lambda_0} \frac{(n\lambda_0)^{k_1}}{k_1!}
\end{equation}

\begin{equation}
P_{H_0} \left( \sum_{i=1}^n X_i > k_1 \right) = \sum_{l = k_1 + 1}^{\infty} \frac{e^{-n\lambda_0} (n\lambda_0)^l}{l!}
\end{equation}

Da discussão anterior, a probabilidade do erro tipo 1 é dada por:
\begin{equation}
\alpha = \delta \, P_{H_0} \left( \sum_{i=1}^n X_i = k_1 \right) + P_{H_0} \left( \sum_{i=1}^n X_i > k_1 \right)
\end{equation}

\textbf{Resumo:}

\textbf{Contexto:} $X_1, \ldots, X_n$ a.c. tal que $X_i \sim \text{Poisson}(\lambda)$

\textbf{Hipóteses:}
\[
\begin{cases}
H_0: \lambda = \lambda_0 \\
H_1: \lambda = \lambda_1
\end{cases}
\]

\textbf{Estatística de teste:}
\[
T(X) = \left( \sum_{i=1}^n X_i \right) \sim \text{Poisson}(n\lambda)
\]

\textbf{Função crítica:}
\[
\psi(T) =
\begin{cases}
1, & \text{se } T > k_1, \\
\delta, & \text{se } T = k_1, \\
0, & \text{se } T < k_1
\end{cases}
\]
