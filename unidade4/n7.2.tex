\begin{equation}
\psi(x^n) = 
\begin{cases}
1, & \text{se } L(\theta_1, x^n) > k L(\theta_0, x^n) \\
0, & \text{se } L(\theta_1, x^n) < k L(\theta_0, x^n)
\end{cases}
\end{equation}

em que $k (\geq 0)$ é determinado por
\begin{equation}
E_{\theta_0} \left[ \psi(x^n) \right] = \alpha
\end{equation}

Qualquer teste satisfazendo (1) e (2) é um teste MP de nível $\alpha$.

\textbf{Dem:} Considere a prova para o caso contínuo. Note que qualquer teste $\gamma$ que satisfaz (2) tem tamanho $\alpha$ e, portanto, nível $\alpha$. Seja $\gamma^*$ um teste com função de teste $\psi_{\gamma^*}(x^n)$ e nível $\alpha$. Sejam $Q_{\gamma}(\theta)$ e $Q_{\gamma^*}(\theta)$ as funções poder de $\gamma$ e $\gamma^*$, respectivamente.

Vamos primeiramente verificar que
\begin{equation}
\left[ \psi_{\gamma}(x^n) - \psi_{\gamma^*}(x^n) \right] \left[ L(\theta_1, x^n) - k L(\theta_0, x^n) \right] \geq 0
\end{equation}
para todo $x \in X^n$. Note que

(i) Se $\psi_{\gamma}(x^n) = 1$, então
\begin{equation}
L(\theta_1, x^n) - k L(\theta_0, x^n) > 0
\end{equation}
