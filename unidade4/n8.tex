De (1) e
\begin{equation}
    \psi_Y(x^1) - \psi_{Y^c}(x^1) \leq 0
\end{equation}

Da definição da função crítica, tentamos (3) seguir.  
Se $\psi_Y(x^1) = 0$, então
\begin{equation}
    l(\theta_i; x^1) - k l(\theta_i; x^1) < 0
\end{equation}

De (1) e
\begin{equation}
    \psi_Y(x^1) - \psi_{Y^c}(x^1) \leq 0
\end{equation}

Decorre da definição de $\psi_Y(x^1)$. Tentamos (3) seguir.  
Se $\alpha \psi_Y(x^1) < 1$, então
\begin{equation}
    l(\theta_i; x^1) - k l(\theta_i; x^1) = 0
\end{equation}

(3) se verifica.  

Daí, tem-se
\begin{equation}
\begin{aligned}
    \alpha \int_{x^n} \{ \psi_Y(x^1) - \psi_{Y^c}(x^1) \} \{ l(\theta_i; x^1) - k l(\theta_i; x^1) \} dx \\
    - \int_{x^n} \psi_Y(x^1) \{ l(\theta_i; x^1) - k l(\theta_i; x^1) \} dx \\
    - \int_{x^n} \psi_{Y^c}(x^1) \{ l(\theta_i; x^1) - k l(\theta_i; x^1) \} dx
\end{aligned}
\end{equation}
