$H_0: \theta = \theta_0 \quad \times \quad H_1: \theta = \theta_1$ para valores grandes de $\frac{L_{1}}{L_{0}}$

\begin{equation}
\frac{L(\theta_1; x)}{L(\theta_0; x)} = \frac{g(T(x); \theta_1)}{g(T(x); \theta_0)}
\end{equation}

que implica que a rejeição de $H_0$ também acontece se, e só se,

\begin{equation}
\frac{g(T(x); \theta_1)}{g(T(x); \theta_0)}
\end{equation}

assume um valor grande.

O LNP também pode ser utilizado para comparar distribuições com densidades distintas.

\textbf{Exemplo 8:} Seja $X$ uma a.a. com densidade f(x) para $x \in \mathbb{R}$. Considere outra densidade:

\[
f_0(x) =
\begin{cases}
\frac{3}{64} x^5, & 0 < x < 4, \\
0, & \text{c.c.}
\end{cases}
\]

\[
f_1(x) =
\begin{cases}
\frac{2}{16} \sqrt{x}, & 0 < x < 4, \\
0, & \text{c.c.}
\end{cases}
\]

Determine o teste MP de nível $\alpha$ para ...
