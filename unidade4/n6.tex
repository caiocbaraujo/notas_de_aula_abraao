\noindent
\underline{chamado de nível $\alpha$ se}
\begin{equation}
    \sup_{\theta \in \Theta_0} \left[ Q_Y(\theta) \right] = \alpha.
\end{equation}

\noindent
Ou, equivalentemente, $Y$ é um teste de tamanho $\alpha$ se
\begin{equation}
    \sup_{\theta \in \Theta_0} E_{\theta} \left[ \psi(X) \right] = \alpha
\end{equation}
ou de nível $\alpha$ se
\begin{equation}
    \sup_{\theta \in \Theta_0} E_{\theta} \left[ \psi(X) \right] \leq \alpha.
\end{equation}

\noindent
A escolha do melhor teste se dará entre aqueles de nível $\alpha$.

\medskip
\noindent
\textbf{Def. 6:} Considere uma classe $\mathcal{C}$ de todos os testes de nível $\alpha$ para $H_0$ e $H_1$. Um teste $Y \in \mathcal{C}$ com função poder $Q_Y(\theta)$ é o melhor teste de nível $\alpha$ ou o teste uniformemente mais poderoso (UMP) de nível $\alpha$ se, e só se,
\begin{equation}
    Q_Y(\theta) \leq Q_{Y^*}(\theta), \quad \forall \theta \in \Theta_1,
\end{equation}
em que $Y^* \in \mathcal{C}$ com função poder $Q_{Y^*}(\theta)$.
