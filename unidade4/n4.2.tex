\begin{equation}
\Phi(z) = \int_{-\infty}^{z} \frac{1}{\sqrt{2\pi}} e^{-t^{2}/2} \, dt
\end{equation}

É a fda da normal padrão.

Outro conceito que será usado em discussão futura é o de função crítica.

\textbf{Def. 4:} A função $\psi_{\gamma}: \chi^{n} \to [0,1]$ é chamada de \textit{função crítica} em \textit{função de teste} $\chi$, e, só se, $\psi_{\gamma}(X)$ representa a probabilidade com a qual $H_{0}$ é rejeitada quando $[X = x]$ é observada.

\textbf{Obs.:} Note que
\begin{equation}
Q_{\gamma}(\theta) = E_{\theta} \left[ \psi_{\gamma}(X) \right], \quad \theta \in \Theta.
\end{equation}

Os testes podem ser classificados como ``aleatorizado'' e ``não aleatorizado''.

\textbf{Def. 5 (Tipos de testes):} Um teste $\gamma$ para hipótese $H_{0}$ pode ser: seja $(x_{1}, \ldots, x_{n})^{T}$ uma realização de $(X_{1}, \ldots, X_{n})^{T}$:
\begin{itemize}
    \item[a)] $\gamma$ não aleatorizado: rejeita $H_{0}$ se, e só se, $(x_{1}, \ldots, x_{n}) \in R$ ou tem função crítica.
\end{itemize}
