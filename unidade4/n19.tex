Da equação: Para $Z \sim N(0,1)$,
\begin{equation}
P(Z > z_{\alpha}) = \alpha \implies 1 - \Phi(z_{\alpha}) = \alpha \ \therefore \ \Phi(z_{\alpha}) = 1 - \alpha
\end{equation}

\begin{center}
\begin{tikzpicture}[scale=1.2]
\draw[->] (-3.5,0) -- (3.5,0) node[right] {$z$};
\draw[->] (0,0) -- (0,1.5) node[above] {$\Phi(z)$};

\draw[domain=-3:3,smooth,variable=\x,black] plot ({\x},{1.2*exp(-\x*\x/2)});

\draw[pattern=north east lines, pattern color=gray] (1.1,0) -- plot[domain=1.1:3] ({\x},{1.2*exp(-\x*\x/2)}) -- (3,0) -- cycle;

\node at (1.2,-0.2) {$z_{\alpha}$};
\end{tikzpicture}
\end{center}

\textbf{Exemplo: (Teste Z)}

\textbf{Suposição:} $X_1, \ldots, X_n \overset{i.i.d.}{\sim} N(\mu, \sigma^2)$ e $\sigma^2$ é conhecida.

\textbf{Hipóteses:}
\[
\begin{cases}
H_0: \mu = \mu_0 \\
H_1: \mu = \mu_1 (> \mu_0)
\end{cases}
\]

\textbf{Estatística de teste:}
\begin{equation}
Z(x_1) = \sqrt{n} \left( \frac{\bar{X} - \mu_0}{\sigma} \right) \ \overset{H_0}{\sim} N(0,1)
\end{equation}

\textbf{Regra de decisão:} (\textit{método tradicional}) Dada uma amostra $x = (x_1, \ldots, x_n)^T$, rejeitamos $H_0$ se $Z(x_1) > z_{\alpha}$.

(\textit{Método do valor-p}): seja $z_{cal} = Z(x_1)$. Rejeitamos $H_0$ se
\begin{equation}
\hat{\alpha} = P(Z > z_{cal}) \leq \alpha
\end{equation}

Obs.: $\hat{\alpha}$ é chamado de valor-p.
