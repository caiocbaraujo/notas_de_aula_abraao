\textbf{Exemplo 7} Seja $X_1, \ldots, X_n$ uma a.a. de $X_i \sim \text{Poisson}(\lambda)$ para $\lambda > 0$ desconhecido. Derive o teste MP para
\[
H_0: \lambda = \lambda_0 \quad \text{e} \quad H_1: \lambda = \lambda_1 \ (\lambda_1 > \lambda_0).
\]

\noindent Id.: como as hipóteses são simples, o LNP se aplica. A verossimilhança é dada por:
\begin{equation}
L_{X} \triangleq L(\lambda; x) = \prod_{k=1}^{n} \left\{ \frac{e^{-\lambda} \lambda^{x_k}}{x_k!} \right\}
= \frac{e^{-n\lambda} \lambda^{\sum_{k=1}^{n} x_k}}{\prod_{k=1}^{n} x_k!}
\end{equation}
em que $x_k \in \{0,1,\ldots\}$. O teste MP é da forma:

\noindent Rejeitar $H_0$ se, e só se, $\frac{L_1}{L_0} > k$.

\noindent Note que:
\begin{equation}
\frac{L_1}{L_0} = \left[ \frac{\lambda_0}{\lambda_1} \right]^n \cdot \left[ \frac{\lambda_1}{\lambda_0} \right]^{\sum_{k=1}^{n} x_k}
= \left[ \frac{\lambda_1}{\lambda_0} \right]^{\sum_{k=1}^{n} x_k - n(\lambda_1 - \lambda_0)}
\end{equation}

\noindent Daí, a região crítica do teste é dada por: $x \in \mathbb{X} \subset \mathbb{Z}^n$, o espaço amostral
\begin{equation}
R_c = \left\{ x \in \mathbb{X} : \left[ \frac{\lambda_1}{\lambda_0} \right]^{\sum_{i=1}^{n} x_i - n(\lambda_1 - \lambda_0)} > k \right\} \Rightarrow
\end{equation}
