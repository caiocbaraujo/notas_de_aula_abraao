\section*{Aula 19 (19/05/2025)}

\subsection*{Unidade 4}

\textbf{Introdução:}

Seja $X$ uma v.a. populacional com fdp (ou fmp) $f(x; \theta)$ para $x \in \mathcal{X} \subset \mathbb{R}$ e $\theta \in \Theta \subset \mathbb{R}$.

\textbf{Def.}: Uma hipótese é uma afirmação sobre o parâmetro desconhecido $\theta$.

Por exemplo:
\begin{equation}
H_1: \mu = \mu_0, \quad H_1: \sigma^2 > \sigma_0^2, \quad H_2: \alpha \neq \alpha_0.
\end{equation}

Neyman e Pearson formularam o problema de testar hipóteses como segue. Considere que se tenha escolher entre:
\begin{equation}
\begin{cases}
H_0: \theta \in \Theta_0, \\
H_1: \theta \in \Theta_1,
\end{cases}
\end{equation}
tal que $\Theta = \Theta_0 \cup \Theta_1$ e $\Theta_0 \cap \Theta_1 = \varnothing$. 

Então, baseando-se numa a.a. $X_1, \ldots, X_n$ de $X$, deve-se tomar a decisão de \underline{rejeitar $H_0$} ou \underline{não rejeitar $H_0$}.

As hipóteses costumam ser classificadas como:
