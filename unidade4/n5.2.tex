$H_0$: $t \leq 50\%$ como ``Rejeita-se $H_0$ se, e só se, \(\sum_{i=1}^n t_i > 5\).''

\(Y\) é aleatorizado com função crítica:

\begin{equation}
\psi_Y(x) =
\begin{cases}
1, & x \in \{3 \in \mathbb{X} \subset \{0,1\}^n : \sum_{i=1}^n z_i > 5 \} \\
\delta, & x \in \{3 \in \mathbb{X} \subset \{0,1\}^n : \sum_{i=1}^n z_i = 5 \} \\
0, & x \in \{3 \in \mathbb{X} \subset \{0,1\}^n : \sum_{i=1}^n z_i < 5 \}
\end{cases}
\end{equation}

Caso \(\psi_Y(x) = \delta \in (0,1)\) a decisão pela rejeição de \(H_0\) se dará por obter ``cara'' no lançamento de uma moeda.

Aula 21 (27/05/2025)

\section*{O conceito de melhor teste}

Considere testar:
\[
H_0 : \theta \in \Theta_0 \quad \text{e} \quad H_1 : \theta \in \Theta_1
\]
tal que \(\Theta = \Theta_0 \cup \Theta_1\) e \(\Theta_0 \cap \Theta_1 = \varnothing\).

\textbf{Definição:} Seja \(\alpha \in (0,1)\) um valor fixado. Um teste \(Y\) para \(H_0\) e \(H_1\), com função poder \(Q_Y(\theta)\), é chamado de \emph{tamanho} \(\alpha\) se
\begin{equation}
\sup_{\theta \in \Theta_0} Q_Y(\theta) = \alpha
\end{equation}
