\section*{Estatística de teste}

\begin{equation}
S = S(\mathbf{x}) = \sum_{i=1}^{n} x_i \quad \text{sob } H_0 \quad \sim \text{Binomial}(n, p_0)
\end{equation}

\section*{Função crítica}

\begin{equation}
\psi(\mathbf{x}) =
\begin{cases}
1, & \text{se } S(\mathbf{x}) > k_1, \\
\delta, & \text{se } S(\mathbf{x}) = k_1, \\
0, & \text{se } S(\mathbf{x}) < k_1,
\end{cases}
\end{equation}

em que $k_1$ é o menor inteiro tal que
\begin{equation}
P\left( S(\mathbf{x}) > k_1 \right) < \alpha
\end{equation}

\begin{equation}
\delta = \frac{\alpha - P_{H_0}\left( S(\mathbf{x}) > k_1 \right)}{P_{H_0}\left( S(\mathbf{x}) = k_1 \right)}
\end{equation}

\section*{Regra de decisão}

\underline{Método tradicional}: Dado $\mathbf{x}$, se $\psi(\mathbf{x}) = 1$, rejeita-se $H_0$.

\underline{Método do valor-p}: Seja $S_{\text{cal}} = S(\mathbf{x})$.

\begin{equation}
\hat{\alpha} = \delta \cdot P_{H_0}\left( S(\mathbf{x}) = S_{\text{cal}} \right) + P_{H_0}\left( S(\mathbf{x}) > S_{\text{cal}} \right)
\end{equation}

Se $\hat{\alpha} < \alpha$, rejeita-se $H_0$.
