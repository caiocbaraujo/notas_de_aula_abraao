\[
R_c = \left\{ x \in \mathcal{X} : \frac{2}{\theta_0} \sum_{i=1}^n x_i > u_\alpha \right\}
\]

Definamos a função \( Q: \mathcal{X} \to \mathbb{R} \) tal que
\[
Q(x) = \frac{2}{\theta_0} \sum_{i=1}^n x_i
\]
Note que \( Q(x) \xrightarrow{H_0} \chi^2_{2n} \). A região crítica fica definida como
\[
R_c = \left\{ x \in \mathcal{X} : Q(x) > q_\alpha \right\},
\]
em que \( q_\alpha \) é o quantil \((1-\alpha)\%\) de \( Q \sim \chi^2_{2n} \), obtido da equação
\[
P(Q > q_\alpha) = \alpha
\]

\begin{center}
\begin{tikzpicture}[scale=1.2]
\draw[->] (-0.5,0) -- (5,0) node[right] {};
\draw[->] (0,-0.5) -- (0,2) node[above] {};
\draw[domain=0:4.5,smooth,variable=\x] plot ({\x},{1.5*\x*exp(-0.5*\x)});
\draw[dashed] (3.5,0) -- (3.5,0.8);
\draw[pattern=north east lines, pattern color=gray] (3.5,0) -- (3.5,0.8) -- (4.5,0.4) -- (4.5,0) -- cycle;
\node at (3.5,-0.3) {$q_\alpha$};
\node at (4.7,0.7) {$\alpha$};
\end{tikzpicture}
\end{center}

\textbf{Resumo (Teste $\chi^2$)}\\
Exponencial: \( X_1, \ldots, X_n \stackrel{\text{i.i.d.}}{\sim} \mathrm{Exp}(\theta) \)\\
Hipóteses:
\[
\begin{cases}
H_0: \theta = \theta_0 \\
H_1: \theta \neq \theta_0 \ (\text{ou } \theta > \theta_0)
\end{cases}
\]
