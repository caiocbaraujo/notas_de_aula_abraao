Um que o inteiro positivo $k_i$, e $g \in (0,1)$ são escolhidos dos tais que o teste tem tamanho $\alpha$. Note que, sob $H_0$,
\begin{equation}
    \sum_{i=1}^n k_i \sim \text{Binomial}(n, p_0)
\end{equation}

Primeiramente, determine o menor inteiro $u_1$ tal que
\begin{equation}
    p_{p_0} \left( \sum_{i=1}^n k_i = u_1 \right) < \alpha, \quad g = \frac{\alpha - p_{p_0} \left( \sum_{i=1}^n k_i > u_1 \right)}{p_{p_0} \left( \sum_{i=1}^n k_i = u_1 \right)}
\end{equation}
em que
\begin{equation}
    p_{p_0} \left( \sum_{i=1}^n k_i = k_1 \right) = \binom{n}{u_1} p_0^{u_1} (1 - p_0)^{n - u_1}
\end{equation}
\begin{equation}
    p_{p_0} \left( \sum_{i=1}^n k_i > u_1 \right) = \sum_{l = u_1 + 1}^n \binom{n}{l} p_0^l (1 - p_0)^{n - l}
\end{equation}

Da discussão anterior, a probabilidade do erro tipo I é dada por
\begin{equation}
    \alpha = g \cdot p_{p_0} \left( \sum_{i=1}^n k_i = u_1 \right) + p_{p_0} \left( \sum_{i=1}^n k_i > u_1 \right)
\end{equation}

Contexto: $k_1, \ldots, k_n \overset{a.o.}{\sim} X \in \mathbb{A}$, em que $X \sim \text{Bernoulli}(p_1)$

Hipóteses:
\begin{equation}
\begin{cases}
H_0: p = p_0 \\
H_1: p = p_1 \ (> p_0)
\end{cases}
\end{equation}
