\begin{equation}
L_i \triangleq L(\theta_i, x) = \prod_{k=1}^n f(x_k, \theta_i), \quad i = 0, 1.
\end{equation}

Considere comparar o poder de todos os testes de nível $\alpha$, com $\alpha$ fixado em $(0,1)$, comumente escolhendo-se valores $\alpha = 1\%, 5\%, 10\%$.

De modo intuitivo, um teste $H_0$ e $H_1$ procura comparar $L_0$ com $L_1$ e procurar qual quantidade é superior à outra. Com critério, a hipótese com verossimilhança mais significativa é favorecida como a mais razoável.

\textbf{Teorema (Lema de Neyman-Pearson)}

Seja $X$ um teste para
\[
H_0 : \theta = \theta_0 \quad \text{e} \quad H_1 : \theta = \theta_1,
\]
com região de rejeição e não rejeição de $H_0$ dadas por:
\begin{equation}
R_c = \left\{ x \in \mathbb{R}^n : L(\theta_1, x) > k \, L(\theta_0, x) \right\}
\end{equation}
\begin{equation}
R_c^c = \left\{ x \in \mathbb{R}^n : L(\theta_1, x) < k \, L(\theta_0, x) \right\}
\end{equation}

ou, equivalentemente, com função crítica dada por...
