\begin{tikzpicture}[scale=1]
    % Axes
    \draw[->] (0,0) -- (6,0) node[right] {$\theta_j$};
    \draw[->] (0,0) -- (0,4) node[above] {$Q(\theta_j)$};
    
    % Horizontal dashed line
    \draw[dashed] (0,3.5) -- (6,3.5);
    
    % Curve
    \draw[thick, domain=1:5, smooth] plot (\x,{1/(1+exp(-2*(\x-3)))+1});
    
    % Point at theta_j = 1.5
    \draw (1.5,0.05) -- (1.5,-0.05) node[below] {15};
\end{tikzpicture}

A probabilidade dos erros do tipo I e II são dadas por:

\begin{equation}
\alpha = P(\text{Erro tipo I}) = P_{H_0}(\text{Rejeitar } H_0) = P_{H_0}(X \in R_C)
\end{equation}

\begin{equation}
\beta = P(\text{Erro tipo II}) = P_{H_1}(\text{Não rejeitar } H_0) = P_{H_1}(X \notin R_C)
\end{equation}

quando se quer testar

\begin{equation}
H_0 : t \in \Theta_0 \quad \text{e} \quad H_1 : t \in \Theta_1
\end{equation}

tal que $H_0$ é chamada de hipótese nula e $H_1$ é chamada de hipótese alternativa.

\textbf{Exemplo:} Sejam $X_1, \ldots, X_n$ uma amostra de $X \sim N(\theta, 1)$ para $\theta \in \mathbb{R}$ desconhecido. Considera que se deseja testar...
