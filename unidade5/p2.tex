\begin{equation}
P_{\alpha}(\mu \in I_{\alpha}) = P_{\mu} \left( \bar{X} - \frac{1.96}{\sqrt{\nu}} < \mu < \bar{X} + \frac{1.96}{\sqrt{\nu}} \right)
\end{equation}

\begin{equation}
= P_{\mu} \left[ \frac{\bar{X} - \mu}{\sqrt{\nu}} < 1.96 \right] \cap \left[ \frac{\bar{X} - \mu}{\sqrt{\nu}} > -1.96 \right]
\end{equation}

\begin{equation}
= P_{\mu} \left( \left| \frac{\bar{X} - \mu}{\sqrt{\nu}} \right| < 1.96 \right)
\end{equation}

\begin{equation}
= P_{\mu} \left( |Z| < 1.96 \right) = 95\%
\end{equation}

\textbf{Aula 21 (25/06/2025)}

Para construção de intervalos de confiança, podem-se utilizar duas abordagens: (i) inversão do procedimento de teste de hipótese e (ii) usando quantidade pivotal.

\textbf{Inversão de um procedimento de teste}

Em teste de hipótese, a região de não rejeição de $H_0$ foi denotada como

\begin{equation}
R_C^C = 
\begin{cases}
\{ x \in \mathcal{X}^n; \ T(x|\theta) \leq k \}^C & \text{para } H_1: \theta > \theta_0, \\
\{ x \in \mathcal{X}^n; \ T(x|\theta) \geq k \}^C & \text{para } H_1: \theta < \theta_0, \\
\text{(como uma solução plausível)} \\
\{ x \in \mathcal{X}^n; \ |T(x|\theta)| \leq k \}^C & \text{para } H_1: \theta \neq \theta_0.
\end{cases}
\end{equation}

O intervalo de confiança é bastante relacionado com $R_C$.