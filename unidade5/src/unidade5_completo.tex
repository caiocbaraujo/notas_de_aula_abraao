\documentclass[12pt,a4paper]{article}
\usepackage[utf8]{inputenc}
\usepackage[T1]{fontenc}
\usepackage[brazil]{babel}
\usepackage{amsmath, amssymb, amsthm}
\usepackage{geometry}
\geometry{margin=2.5cm}
\usepackage{hyperref}
\hypersetup{colorlinks=true,linkcolor=blue,urlcolor=blue}
\usepackage{import}
\usepackage{tikz}
% TikZ patterns needed by some figures
\usetikzlibrary{patterns}
\usepackage{array}

% Título e informações do documento
\title{Unidade 5 - Compilação Completa\\
\large Intervalos de Confiança e Tópicos Relacionados}
\author{Curso de Inferência Estatística}
\date{Outubro 2025}

\begin{document}

\maketitle
\tableofcontents
\newpage

% Importando todas as páginas da pasta unidade5 (arquivos no diretório pai)
\import{../}{p1.tex}
\import{../}{p1.2.tex}
\import{../}{p2.tex}
\import{../}{p2.2.tex}
\import{../}{p3.tex}
\import{../}{p3.2.tex}
\import{../}{p4.tex}
\import{../}{p4.2.tex}
\import{../}{p5.tex}
\import{../}{p5.2.tex}
\import{../}{p6.tex}
\import{../}{p6.2.tex}
\import{../}{p7.tex}
\import{../}{p7.2.tex}
\import{../}{p8.tex}
\import{../}{p8.2.tex}
\import{../}{p9.tex}
\import{../}{p9.2.tex}
\import{../}{p10.tex}
\import{../}{p10.2.tex}
\import{../}{p11.tex}
\import{../}{p11.2.tex}
\import{../}{p12.tex}
\import{../}{p12.2.tex}
\import{../}{p13.tex}
\import{../}{p13.2.tex}
\import{../}{p14.tex}
\import{../}{p14.2.tex}
\import{../}{p15.tex}
\import{../}{p15.2.tex}
\import{../}{p16.tex}
\import{../}{p16.2.tex}
\import{../}{p17.tex}
\import{../}{p17.2.tex}
\import{../}{p18.tex}
\import{../}{p18.2.tex}
\import{../}{p19.tex}
\import{../}{p20.tex}
\import{../}{p20.2.tex}
\import{../}{p21.tex}
\import{../}{p21.2.tex}

\end{document}

