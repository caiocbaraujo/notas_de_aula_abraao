Sup $2L(\hat{\theta})$ em comparação com valores de $2L(\theta)$.

Note que o $c_{1\alpha}$ e $c$ deve ser definido em $(0,1)$ tal que o TSU tenha nível $\alpha$.

Seguem duas notas importantes:

(1) No entanto, com frequência se objetiva testar parte dos parâmetros de $\theta$, digamos $\theta_{0} = (\theta_{1}, \ldots, \theta_{q})^{T}$ tal que $q < p$, conhecidos como parâmetros de interesse:

\[
H_{0} : (\theta_{1}, \ldots, \theta_{q}) = (\theta_{1,0}, \ldots, \theta_{q,0})
\]

Os demais parâmetros $(\theta_{q+1}, \ldots, \theta_{p})$ são chamados de parâmetros de perturbação ou incógnitos.

(2) Sobre a derivação de $\Lambda$:

\begin{equation}
\sup_{\theta \in \Theta_{0}} \{ L(\theta) \} = L(\hat{\theta}),
\end{equation}

em que $\hat{\theta}$ representa o estimador de máxima verossimilhança (EMV) restrito (assumindo o parâmetro de interesse conhecido):

\[
\hat{\theta} = (\theta_{1}, \ldots, \theta_{q}, \hat{\theta}_{q+1}, \ldots, \hat{\theta}_{p})
\]

