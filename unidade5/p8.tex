\newpage

\section*{Aula 29 (02/07/2023)}

Se $\sigma$ é conhecido, podemos obter $a, b$ tais que
\begin{equation}
P_{\theta} \left( a < \frac{\hat{\kappa}(\theta) - \kappa(\theta)}{\sigma} < b \right) = 1 - \alpha
\end{equation}

Desta última identidade, obtém-se o intervalo de confiança $1 - \alpha$ para $\kappa(\theta)$.

Para $\sigma$ desconhecido:
\begin{equation}
P_{\theta} \left( a < \frac{\hat{\kappa}(\theta) - \kappa(\theta)}{\hat{\sigma}} < b \right) = 1 - \alpha
\end{equation}

Desta identidade, obtém-se o intervalo de confiança.

Os resultados anteriores são locação. Quando a inferência é sobre o parâmetro de escala, costuma-se utilizar o pivô
\begin{equation}
U = \frac{\hat{\kappa}(\theta)}{\kappa(\theta)}
\end{equation}
cuja distribuição geralmente independe de $\theta$. Para este caso, usa-se
\begin{equation}
P_{\theta} \left( a < \frac{\hat{\kappa}(\theta)}{\kappa(\theta)} < b \right) = 1 - \alpha
\end{equation}