\section*{TRV para duas amostras}

Nesta seção vamos discutir problemas relacionados ao TRV para duas amostras em dois contextos:

para $X \sim N(\mu_x, \sigma_x^2)$ \quad e \quad $Y \sim N(\mu_y, \sigma_y^2)$

\subsection*{Caso 1:}
\begin{equation}
\begin{cases}
H_0: \mu_x = \mu_y \\
H_1: \mu_x \neq \mu_y
\end{cases}
\quad \text{para} \quad \sigma_x^2 = \sigma_y^2 = \sigma^2
\end{equation}

\subsection*{Caso 2:}
\begin{equation}
\begin{cases}
H_0: \sigma_x^2 = \sigma_y^2 \\
H_1: \sigma_x^2 \neq \sigma_y^2
\end{cases}
\quad \text{para} \quad \mu_x \neq \mu_y
\end{equation}

\section*{TRV para comparar médias}

Sejam $x_1, \ldots, x_n$ e $y_1, \ldots, y_m$ a.a.s independentes de $X \sim N(\mu_x, \sigma_x^2)$ e $Y \sim N(\mu_y, \sigma_y^2)$, respectivamente.

Assuma que $\sigma_x^2 = \sigma_y^2 = \sigma^2 \in \mathbb{R}_+$ e $\theta = (\mu_x, \mu_y, \sigma^2) \in \mathbb{R} \times \mathbb{R} \times \mathbb{R}_+$ é desconhecido. 

Dado $\alpha \in (0,1)$ como o nível, vamos derivar o TRV para
\begin{equation}
H_0: \mu_x = \mu_y \quad \text{vs} \quad H_1: \mu_x \neq \mu_y
\end{equation}

Inicialmente, lembramos que:

