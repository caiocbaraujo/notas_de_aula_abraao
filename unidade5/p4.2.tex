\newpage

\textbf{Exemplo 4:} Seja $X \sim \text{Exp}(\theta)$ com densidade
\begin{equation}
    f(x \mid \theta) = \frac{1}{\theta} e^{-x/\theta} I_{(0,\infty)}(x).
\end{equation}

Encontrar um intervalo de confiança $1 - \alpha$ bilateral para $\theta$.

\textbf{Solução:} Note que $U := \frac{X}{\theta}$ tem densidade
\begin{equation}
    f_U(u) = \frac{d F_U(u)}{du} = \frac{d \mathbb{P}(X \leq u\theta)}{du} = \theta \cdot \frac{1}{\theta} e^{-u} I_{(0,\infty)}(u).
\end{equation}

Logo:
\begin{equation}
    f_U(u) = e^{-u} I_{(0,\infty)}(u).
\end{equation}

Portanto $U$ pode ser entendido como um pivô. Note que é possível definir $a,b \in \mathbb{R}$ com $a < b$ tais que
\begin{equation}
    \mathbb{P}(U < a) = \mathbb{P}(U > b) = \frac{\alpha}{2}.
\end{equation}

E, portanto:
\begin{equation}
    \mathbb{P}(a < U < b) = 1 - \alpha.
\end{equation}

Com $\alpha \in (0,1)$ fixado:

\begin{equation}
    \int_{0}^{a} e^{-u} \, du = 1 - e^{-a} = \frac{\alpha}{2} \quad \Rightarrow \quad e^{-a} = 1 - \frac{\alpha}{2} \quad \Rightarrow \quad a = -\log\left(1 - \frac{\alpha}{2}\right).
\end{equation}

\begin{equation}
    \int_{b}^{\infty} e^{-u} \, du = e^{-b} = \frac{\alpha}{2} \quad \Rightarrow \quad b = -\log\left(\frac{\alpha}{2}\right).
\end{equation}